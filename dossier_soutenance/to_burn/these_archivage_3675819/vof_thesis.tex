In this section we detail only the necessary steps to 
illustrate the momentum advection method based on the VOF method.
To simplify the presentation we rescale space and time variables so 
that the cell volume and the time step are both equal to 1. 
Any velocity component $u$ becomes $u^\prime = u \tau / h$ and 
the $\cijk$ value is also the measure of the volume of reference fluid in cell 
$i,j,k$.
At the beginning of any simulation the volume fraction field is
initialized with the {\sc Vofi} library described in \cite{bna2015numerical} 
and \cite{bna2016vofi}. This allows a highly accurate numerical integration of 
the measure of fluid volumes. 

\subsubsection{Normal vector and plane constant determination} 

The VOF method proceeds in two steps, reconstruction and advection. 
In the first step we consider a PLIC reconstruction in each cell cut by the 
interface where the unit normal vector $\N$ is computed 
with the MYC method described in \cite{Tryggvason11}. We then consider
the colinear normal vector $\M$ whose components satisfy the relation
$|m_x| + |m_y| + |m_z|=1$. Given the
volume $V=\cijk$ in cell $i,j,k$ occupied by fluid 1 and the normal $\M$
we consider the family of planes 
\be 
\M \cdot \X = \alpha \label{mxalpha}
\nd 
By changing the value of the plane constant $\alpha$ a different
volume of fluid 1 is cut in the cell. The correct value of
$\alpha$ is determined by the resolution of a cubic equation \cite{Scardovelli00}.

\subsubsection{General split-direction advection}
\label{generalsplit}
The interface reconstruction at time $t_{n}$ is then used to obtain the  
position of the interface and the volumes $\cijk$ at the next discrete time $t_{n+1}$. 
The following discussion of momentum advection is based on 
two VOF advection methods, Lagrangian Explicit and Weymouth and Yue's 
schemes. We first describe their common features. 

After addition and subtraction of a term proportional to the velocity divergence, 
equation (\ref{interfadv}) leads to 
\be
\dert H + \nabla \cdot (\U H) = H\, \nabla \cdot \U \,.
\nd
This equation is integrated in the time step and cell volume
\be
{\cijk^{n+1} - \cijk^{n}} = - \sum_{\rm{faces}\, f} F^{(c)}_f + \int_{t_n}^{t_{n+1}}  
{\rm d}t \int_\Omega  H \,\nabla \cdot \U\,  {\rm d}\X   \,,
\label{sumf}
\nd
where the first term on the right-hand side is the sum over the cell faces $f$  
of the fluxes $F^{(c)}_f$ of $(\U H)$. Obviously the ``compression'' term on the 
right-hand side disappears for incompressible flow, however it is essential 
in split-advection methods.  In the previous equation the flux $F_f^{(c)}$ is 
\be
F_f^{(c)} = \int_{t_n}^{t_{n+1}} {\rm d}t \int_{f} u_f(\X,t) H(\X,t) \, {\rm d}\X \,,
\label{faceint}
\nd
where $u_f = \U\cdot \N_f$.
Once an approximation for the evolution of $(\U H)$ during
the time step is chosen, a four-dimensional integral remains to be
computed in equation (\ref{faceint}). The two methods we consider here
are directionally split and are also designed to preserve the
property $0 \le \cijk \le 1$ which we call $C$-bracketing. It
is important to preserve $C$-bracketing in order to avoid arbitrary addition or
removal of mass. Furthermore they do not produce in the bulk of the
two fluids deviations from the correct values $0$ and $1$.


Directional splitting results in the breakdown of equation (\ref{sumf}) 
into three equations
\be
{\cijk^{n,l+1} - \cijk^{n,l}} = - F^{(c)}_{m-} - F^{(c)}_{m+} 
+ c_m \partial_{m}^h u_m \,,
\label{sumf2}
\nd
where the superscript $l=0,1,2$ is the substep index, i.e.
$\cijk^{n,0} = \cijk^{n}$ and $\cijk^{n,3} = \cijk^{n+1}$. 
The face with subscript $m-$ is the ``left'' face in direction $m$ with 
$F^{(c)}_{m-} \ge 0$ if the flow is locally from right to left. A similar reasoning 
applies to the ``right'' face $m+$.
 We have also  approximated the compression term in 
(\ref{sumf}) by
\be
 \int_{t_n}^{t_{n+1}}  {\rm d}t \int_\Omega  H \partial_m u_m  {\rm d}\X \simeq  c_m 
 \partial_{m}^h u_m \,, 
 \label{comp}
\nd
with no implicit summation rule. 
 In the RHS of (\ref{sumf2}) and (\ref{comp}) the flux terms $F_{f}^{(c)}$ and the 
 partial derivative $\partial_{m} u_m$ must
be evaluated with the same discretized velocities. In particular,
$\partial_{m}^h u_m$ is a finite difference or finite volume approximation of the 
spatial derivative of the $m$th component of the velocity vector in direction $m$, and
the ``compression coefficient'' $c_m$ approximates the color fraction. 
Its exact expression is dependent on the advection method and 
it preserves the $C$-bracketing condition. 
Since the coefficient $c_m$ may not be the same along the three Cartesian 
directions, the sum $\sum_m c_m \partial_{m}^h u_m$ is not necessarily vanishing
even if the flow is incompressible.
After each advection substep (\ref{sumf2}), the interface is reconstructed
with the updated volumes $\cijk^{n,l+1}$, then the 
fluxes $F^{(c)}_{f}$ are computed for the next substep. 

In each substep the velocity field in direction $m$ is usually dependent
only on the spatial component $x_m$, $u_m (x_m)$. This approximation ensures that the 
fluid areas fluxing across a cell side are rectangular in two dimensions, as shown in
Figs. \ref{lagfig} and \ref{eulflux}. The multi-dimensionality of the flow
is considered in unsplit methods, where the fluxing areas are described by more 
complex polygons \cite{Cervone09}.

\subsubsection{Lagrangian Explicit advection}

The Lagrangian Explicit / CIAM method refers to a specific type
of split advection and it is most naturally explained
as a Lagrangian transport of the Heaviside function \cite{li95,Scardovelli02} . 
The velocity field is linearly interpolated between the face velocities $u_{m-}$ on
the left and $u_{m+}$ on the right, so that 
$u_m (x_m) = - u_{m-}( 1 - x_m) + u_{m+} x_m$, with the origin of
$x_m$ on the left face. 
The equation of motion $d x_m / dt = u_m (x_m)$ is integrated with a
first-order in time, explicit $u_m (x^n_m)$ approximation, to get in rescaled variables 
\be
x_m^{n+1} =  - u_{m-} +  (1 +  u_{m+} + u_{m-}) \,x^n_m \,.
\label{map}
\nd
This transformation gives the position of advected points as a function of the 
original position and compresses distances along direction $m$ by a factor 
$(1 + u_{m+} +  u_{m-})$. Points over faces and linear interface are advected
in the same way, and 
in the two-dimensional case of Fig. \ref{lagfig} the advection substep results 
in the three contributions $V_1$, $V_2$ and $V_3$ to the central cell.
The intermediate value of the color function in the central cell
will be given by the sum of these three contributions.
\begin{figure}
\begin{center}
    \includegraphics[width=\textwidth]{Figures/LE_mapping-horbis.pdf}
\end{center}
\caption{Formation of the areas  $V_1,V_2$ and $V_3$ by Lagrangian advection in the 
horizontal direction: initial reconstruction with the horizontal velocities on the faces 
of the central cell (left); segments and areas $V_i$ after Lagrangian 
advection (right). In three dimensions rectangles become cuboids.}
\label{lagfig}
\end{figure}

There is a correspondence between the geometrical interpretation of the Lagrangian 
Explicit advection and the definition \eqref{faceint} of $F^{(c)}_{f}$. 
For example, for the central cell of Fig. \ref{lagfig} the flux on
the left face is from left to right, since $u_{1;i-1/2,j} > 0$.
Then with $m = 1$ and $f=1-$, we have $V_{1} =  - F^{(c)}_{1-} > 0$ and 
$u_{1-} =\U\cdot \N_{1-} < 0$.
The final expression of the substep is 
\be
\cijk^{n,l+1} =  \cijk^{n,l} (1  +  u_{m+} +  u_{m-} )  - F^{(c)}_{m-} - F^{(c)}_{m+} \,,
\label{sumflag}
\nd
which shows that the approximation of the derivative is 
$\partial_m^h u_m = u_{m+}  + u_{m-}$ and the compression coefficient is
\be
c_m = \cijk^{n,l} 
\label{cmle}
\nd
In order to remove spurious asymmetries in the flow it is important
to change the order of split advections at each timestep. Then
the compression coefficient in the horizontal advection, $m=1$, can
be associated to the first substep, $l=0$, in timestep $n$, and to
the last substep, $l=2$, in the next timestep.
The sequence of three Lagrangian substeps (\ref{sumf2}) 
does not result in volume conservation
\be
{\cijk^{n+1} - \cijk^{n}} = - \sum_{\rm{faces}\, f} F^{(c)}_f 
+ \sum_{m=1}^{3} c_{m}\,(u_{m+}  + u_{m-}) \,.
\label{sumfall}
\nd
While the flux terms cancel upon integration over the domain, the sum of the compressive 
terms does not vanish since $c_m$ changes at each substep $l$.

\begin{figure}
\begin{center}
    \includegraphics[width=\textwidth]{Figures/EI_mapping-horbis}
\end{center}
\caption{Eulerian flux representation for advection in the horizontal direction: 
same initial reconstruction and horizontal velocities of Fig. \ref{lagfig} (left); 
fluxes, or areas $V_1$ and $V_3$, are calculated directly from the interface
reconstruction in each cell (right)}
\label{eulflux}
\end{figure}

\subsubsection{Weymouth and Yue's advection}

The (WY) split advection is exactly mass-conserving  \cite{Weymouth:2010hy}. 
In this method the compression coefficient is independent of direction $m$, 
so that $c_m = c$, and is defined as
\be
c = H \big( \cijk^n - 1/2 \big) 
\label{cmwy}
\nd
where $H$ is a one-dimensional Heaviside function, 
that is $c=0$ if $\cijk^n < 1/2$ and $c=1$ if  $\cijk^n \ge 1/2$. 
The fluxes $F^{(c)}_f$ are also defined differently. 
The reference phase fluxed through the left face in direction $m=1$ 
is equal to the volume fraction in a cuboid of width $u_{i-1/2,j,k}$ 
adjacent to the left face $f=1-$. This fluxed volume 
corresponds to ``Eulerian Implicit'' (EI) advection in the terminology 
of \cite{Scardovelli02} and is represented in 2D by the area $V_1$ of 
Fig. \ref{eulflux}. Using these definitions,  
Weymouth and Yue were able to show that the final result obeys 
$C$-bracketing \cite{Weymouth:2010hy}. 

This split advection scheme conserves volume at machine accuracy. 
Indeed the summation of three substeps (\ref{sumf2}) results in 
\be
{\cijk^{n+1} - \cijk^{n}} = - \sum_{\rm{faces}\, f} F^{(c)}_f 
+ c \sum_{m=1}^{3} \partial_{m}^h u_m \label{sumfall2}
\nd
Since $\sum_{m=1}^{3} \partial_{m}^h u_m = \sum_{m=1}^{3} (u_{m+}  + u_{m-}) $ is 
the finite-volume expression for $\nabla \cdot \U$, it disappears and mass 
is conserved at the 
accuracy with which condition (\ref{divu}) is satisfied. 

\subsubsection{Clipping}
\label{clipping}

The algorithm that has been coded involves a number of additional
steps designed to avoid unwanted effects of arithmetic floating point
round-off error. The most important one is clipping: at the end of
each directional advection, the values of $\cijk$ are reset so that
$C_{ijk}$ is set to $0$ if $C_{ijk} < \eps_c$ or to
$1$ if $C_{ijk} > 1 - \eps_c$. When there is no surface tension 
%with large curvature $\kappa h$ 
the choice $\eps_c = 10^{-12}$ works well.
Otherwise $\eps_c = 10^{-8}$ gives more stable results with smoother
interface shapes. This stronger clipping is a necessity for some
simulations with WY, as we observe that WY produces many more ``wisps'', 
i.e. cells with tiny values of $1-\cijk$
inside fluid 1 or $\cijk$ inside fluid 2.
We have not yet been able to determine the origin of this need for a more
forceful clipping with WY, but it could be related to the fact that the
CIAM method has a geometrical interpretation, while WY is intrisically algebraic
in nature.
