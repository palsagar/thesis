\setchapterpreamble[u]{\margintoc}
\chapter{Numerical Benchmarks}
\labch{numbench}

In the upcoming sections, we demonstrate the robustness and accuracy of our class of mass-momentum consistent numerical methods when applied to challenging high density-ratio flow configurations, primarily in comparison to the version of our method which does not maintain consistency between the mass and momentum advection. Most of the standard tests that exist in the current literature concerning numerical methods to tackle liquid-gas flows such as the decay of spurious currents in static and moving droplets, viscous damping of capillary waves etc., are carried out in the absence of any density jump (or viscosity jump) across the interface separating the fluids. In this chapter, we shall take a closer look in detail at the behavior of our methods when dealing with difficulties that arise due to the non-linear coupling between interfacial deformation/propagation, capillary and viscous forces, especially in the regime where the material properties across the interface are separated by orders of magnitude, particularly in which the flow features in question are poorly resolved. 

In order to assess the performance of the different methods, we shall use an easier nomenclature to describe the different methods, which are as follows : 

\begin{itemize}
	\item \textbf{M1} Method with non-consistent momentum-mass transport.
	\item \textbf{M2} Method with consistent momentum-mass transport, but not conservative. Uses half-fractions strategy. 
	\item \textbf{M3} Method with consistent and conservative momentum-mass transport. Uses sub-grid strategy. 
\end{itemize}


\section{Static Droplet}
\labsec{static}

A popular numerical benchmark in the existing literature relevant to surface tension dominated flows is the case of a spherical droplet of the denser fluid immersed in a quiescent surrounding medium of the lighter fluid. In the hydrostatic limit of the Navier-Stokes equations, the droplet should stay in equilibrium, with a curvature induced pressure jump across the interface corresponding to Laplace's equilibrium. In practice however, numerically reproducing such a trivial equilibrium condition is not as straighforward, as there exists a slight difference between the initial numerical interface and the exact analytical shape of the sphere, thereby resulting in the generation of the well documented '\textit{spurious}' or '\textit{parasitic}' currents of varying intensity in the velocity field \sidecite{lafaurie1994modelling, harvie2006analysis, popinet1999front}. A lot or progress has been made since in the context of \textit{well-balanced} surface tension formulations, that ensure consistency between the numerical stencils used for the discretization of the pressure gradient and the Heaviside approximation ($n \delta_{s}$) that projects the the surface force distribution onto the control volumes \cite{francois2006balanced,popinet2009accurate}. A significant contribution to the interpretation of these parasitic currents within the well-balanced framework was made by Popinet \cite{popinet2009accurate} which demonstrated that given sufficient time (of the order of viscous dissipation time-scales), a well-balanced method will relax to the '\textit{numerical}' equilibrium shape through the damping of the 'physically consistent' numerical capillary waves, therefore allowing us to recover the exact (to machine precision) Laplace equilibrium condition.

\subsection*{Setup}

% insert figure  

\begin{figure}[h!]
    \centering
    \includegraphics[width = 0.8\textwidth]{plots/static_drop/config.png}
    \caption{Schematic of the static droplet of dense fluid surrounded by a quiescent medium of lighter fluid. A $40 \times 40$ grid is employed to spatially discretize the domain.}
    \label{static_conf}
\end{figure}

% difference in our setup in terms of high-density ratios
The key difference in our implementation of this classic test case from that of Popinet \sidecite{popinet2009accurate} is that we consider the effect of density contrast across the interface separating the fluids. As we have previously discussed, a sharp density jump across the interface has an amplification effect on the numerical errors incurred as a result of interfacial reconstructions, curvature estimation and various other truncations, thereby rendering the method unstable. We demonstrate that in our framework of mass consistent momentum transport coupled with a well-balanced surface tension discretization, density-ratios as large as $1000:1$ can be simulated without loss of numerical stability, in conjuction with the ability to recover the exact numerical equilibrium through the dissipation of spurious currents within relevant time-scales \sidenote{The viscous time-scale corresponding to the droplet length-scale is the most commonly used in literature.}.

% specification of numerical problem setup, domain, densities and viscosities
We consider a circular droplet of size $D$  placed at the centre of a square domain of side $L$. The densities of the heavier and lighter phases are $\rho_l$ and $\rho_g$ respectively, likewise for the viscosities $\mu_l$ and $\mu_g$, and $\sigma$ being the surface tension coefficient (fig. \ref{static_conf}). The ratio of the droplet size to the box is chosen as $D/L = 0.4$, coupled with a numerical resolution of $D/\Delta x= 16$ (where $\Delta x$ is the grid size). As for boundary conditions, we use symmetry conditions on all sides of the square domain.

% specification of problem parameters and adimensional numbers 
The problem incorporates two natural time-scales, the capillary oscillation scale and the viscous dissipation scale, which are defined below :

\begin{align}
        T_\sigma = \left(\frac{\rho_l D^3}{\sigma}\right)^{1/2} \quad , \quad T_\mu = \frac{\rho_l D^2}{\mu_l}
\label{ts}
\end{align}

The ratio of these time-scales give us -

\begin{align}
        \frac{T_\mu}{T_\sigma} = \sqrt{\rho_l \sigma D}/\mu_l = \sqrt{La}
\end{align}

where $La$ is the Laplace number based upon the heavier fluid. In the present study, we introduce the density-ratio $\rho_l/\rho_g$ as another important parameter. In order to rescale our 'parasitic' velocity field, we define a velocity scale based on capillary oscillations as -

\begin{align}
        U_\sigma = \sqrt{\sigma/\rho_l D}
\end{align}

Additionally, the time-step in our numerical simulation must be smaller than the oscillation period corresponding to the grid wavenumber (fastest capillary wave with a time period $\sim \left( \rho_l \Delta x^3 / \sigma  \right)^{1/2} $ ) as a stability criterion \sidenote{Similar criteria are defined on the basis of the viscous and advection operators as well, with the smallest amongst the three selecting the numerical time-step}, as our surface tension model is explicit in time. For the scope of the present study, we shall not consider any viscosity contrast between the two fluids while varying the density-ratio, therefore $\mu_l/\mu_g = 1$ for all the cases under study.


\subsection*{Decay of Spurious Currents}

In figures \ref{decay_nonmc} to \ref{decay_sagar}, we illustrate the decay of the root-mean-square of the spurious currents as a function of time, in the case of four different density-ratios, with three different Laplace numbers for each ratio. The first figure (\ref{decay_nonmc}) refers to simulations carried out without consistency between the momentum-mass transport (M1), the second (\ref{decay_daniel}) corresponds to that of the consistent but not conservative method (M2), and final one (\ref{decay_sagar}) refers to that of the consistent and conservative method (M3). The time is rescaled by the viscous dissipation scale, and the spurious currents by the capillary velocity scale. We have two main observations, the rapid decay of the rescaled spurious currents for all combinations of density-ratios and Laplace numbers within approximately $0.2 T_{\mu}$, and the slower re-growth of the currents in question for combinations of non-unity density-ratios and large Laplace numbers, in all simulations except those carried out with \textbf{M3}. With method \textbf{M3}, the decayed currents keep hovering around levels of machine precision for remainder of time. Although there is a re-growth of the currents using the consistent method (\textbf{M2}) after $0.2 T_{\mu}$, the behavior is not quite alarming as the rate of this re-growth is quite low. Therefore, out of all the methods tested, the consistent and conservative method (\textbf{M3}) does seem to demonstrate the desired performance, especially when it comes to combinations of large density contrasts coupled with large Laplace numbers.   


\begin{figure}[h!]
    \centering
    \includegraphics[]{plots/static_drop/decay_nonmc.png}
	\caption{\textbf{M1} Decay of normalized spurious currents as a function of viscous dissipation time-scales for different density-ratios and Laplace numbers. The currents seem to initially decay quickly for all higher density-ratios, and relax to the numerical equilibrium curvature even within $0.2 \cdot T_\mu$. For combinations of large $\rho_l / \rho_g$ and large $La$, the spurious currents seem to grow back to an order of magnitude ($10^{-8}$) which is quite far from that of machine precision ($10^{-14}$).}   
    \label{decay_nonmc}
\end{figure}

\begin{figure}[h!]
    \centering
    \includegraphics[]{plots/static_drop/decay_daniel.png}
	\caption{\textbf{M2} Decay of normalized spurious currents as a function of viscous dissipation time-scales for different density-ratios and Laplace numbers. The currents seem to initially decay quickly for all higher density-ratios, and relax to the numerical equilibrium curvature even within $0.2 \cdot T_\mu$. For combinations of large $\rho_l / \rho_g$ and large $La$, the spurious currents seem to grow back to an order of magnitude ($10^{-8}$) which is quite far from that of machine precision ($10^{-14}$). No considerable improvement is observed with respect to \textbf{M1}. }   
    \label{decay_daniel}
\end{figure}

\begin{figure}[h!]
    \centering
    \includegraphics[]{plots/static_drop/decay_sagar.png}
	\caption{\textbf{M3} Decay of normalized spurious currents as a function of viscous dissipation time-scales for different density-ratios and Laplace numbers. The currents seem to decay very quickly in the case of higher density-ratios, and relax to the numerical equilibrium curvature even within $0.2 \cdot T_\mu$. For all combinations of $\rho_l / \rho_g$ and $La$ numbers, the decayed spurious currents are not observed to grow back as in the cases of \textbf{M1} and \textbf{M2}, and hover around values close to machine precision ($10^{-14}$).}   
    \label{decay_sagar}
\end{figure}



\subsection*{Spatial Convergence}

Once the solution relaxes to a numerical equilibrium curvature (spurious currents are approximately at the order of machine precision), there still exists a difference between the numerical curvature and the exact analytical curvature corresponding to the spherical (circular) shape. We use the definitions of the shape errors as introduced in the seminal work of Popinet \cite{popinet2009accurate} to assess the convergence of our class of methods to the exact (analytical) curvature as we increase spatial resolution. The norms are defined as follows :      

\begin{align}
	L_2 = \sqrt{\frac{\sum_i \left(C_i - C_i^\text{exact} \right)^2}{\sum_i}} \quad , \quad L_\infty = \text{max}_i \left( | C_i - C_i^\text{exact} | \right)
  \label{shape_err_norms}
\end{align}

where $C_i$ is the volume fraction of a cell after the solution has relaxed to the numerical equilibrium curvature, and $C_i^\text{exact}$ is the volume fraction corresponding to the exact circular shape which was initialized at the start of the simulation.  

Fig. \ref{static_drop_conv} demonstrates the behavior of the shape errors defined in eqn. \ref{shape_err_norms} for the case of the most stringent parameter combination ( $\rho_l / \rho_g = 1000 $ , $La = 12000$ ) as a function of the droplet resolution. As one can clearly observe, all the methods tested display a roughly second-order convergence in space for both the error norms. In terms of the $L_2$ norm, the consistent and conservative method (\textbf{M3}) does indeed achieve smaller errors as compared to both \textbf{M1} and \textbf{M2} for all spatial resolutions. As a minor remark, there is not much to discern in terms of shape error when it comes to comparing the performances of the consistent (\textbf{M2}) method with the non-consistent one (\textbf{M1}). 

\begin{figure}[h!]
    \centering
    \includegraphics[width = 0.9\textwidth]{plots/static_drop/static_drop_convergence.png}
	\caption{Second-order spatial convergence for the spurious current error norms corresponding to the most stringent parameter combination ($\rho_l/\rho_g = 1000$ , $La = 12000$) . Both of the norms ($L_\infty$ and $L_2$) seem to demonstrate a roughly second order rate of spatial convergence with each of the methods tested. However, \textbf{M3} has a marginally lower $L_2$ error compared to both \textbf{M1} and \textbf{M2} for all resolutions tested. There is negligible difference observed in the shape errors between \textbf{M1} and \textbf{M2} in both of the norm definitions.}   
    \label{static_drop_conv}
\end{figure}



%---------------------------------------------------------------

\section{Moving Droplet}

An incisive numerical setup which evaluates the accuracy of the coupling between interfacial propagation and surface tension discretization was first proposed by Popinet \cite{popinet2009accurate}, subsequently employed in the comparative study of Abadie et al. \sidecite{abadie2015combined}. The manner in which this test differs from that of the static droplet is the presence of a uniform background velocity field, therefore serving as a better representation of droplets in complex surface tension dominated flows where they might be advected by the mean flow.

\subsection*{Setup}

\begin{figure}[h!]
    \centering
    \includegraphics[width = 0.8\textwidth]{plots/droplet_advect/config.png}
    \caption{Schematic of the droplet of dense fluid advected in a surrounding medium of lighter fluid. A $50 \times 50$ grid is employed to spatially discretize the domain, which is spatially periodic in the direction of droplet advection.}
    \label{moving_conf}
\end{figure}

In terms of the Laplace equilibrium, the hydrostatic solution is still valid in the frame of reference of the moving droplet. The point at which the solution in the moving reference frame diverges from that of the static droplet (\ref{sec:static}) is through the continuous injection of noise at the scale of the grid size. This 'numerical' noise emanates from the perturbations to the curvature estimates, which are in turn induced by the interfacial reconstructions carried out to propagate the interface. These fluctuating errors act as source terms for the momentum, thereby transforming the problem into that of viscous dissipation in the presence of continuous forcing (in the moving reference frame).

In the present study, we evaluate our method using the advection of a droplet in a spatially periodic domain in the same setup as \cite{popinet2009accurate}, but with the important difference of including sharp density jumps across the interface. As previously discussed (\ref{sec:static}), high density-ratios tend to rapidly amplify the fluctuations induced by the myriad numerical approximations (interface reconstruction, curvature estimation etc) involved in the algorithm, thereby leading to loss of numerical stability.

We again consider a circular droplet of size $D$ placed at the centre of a square domain of side $L$. The densities of the heavier and lighter phases are $\rho_l$ and $\rho_g$ respectively, likewise for the viscosities $\mu_l$ and $\mu_g$, and $\sigma$ being the surface tension coefficient (fig. \ref{static_conf}). A uniform velocity field $U$ is initialized on the entire domain (horizontal direction). The ratio of the droplet size to the box is $D/L = 0.4$, with $D/\Delta x= 20$ ($\Delta x$ being the grid size). As for boundary conditions, we use symmetry conditions on the top and bottom sides, and periodic boundary conditions on the horizontal direction (along which advection by $U$ takes place).

We characterize by problem by introducing the following adimensional numbers (based on the heavier fluid) :

\begin{align}
        La = \frac{\rho_l \sigma D}{\mu_l^2} \quad , \quad We = \frac{\rho_l U^2 D}{\sigma}
\end{align}

In addition to the capillary and viscous time-scales for the static case (eqns. \ref{ts}), we have an additional scale defined as :

\begin{align}
	T_{u} = D/U
\end{align}

which is the time-scale of advection. In our subsequent analysis, we shall use $T_u$ and $U$ as the time and velocity scales, repectively.

\subsection*{Evolution of Spurious Currents}

Figures \ref{evo_nonmc} tp \ref{evo_sagar} depict the evolution of the root-mean-square (RMS) error of the velocity field in the moving frame of reference, as a function of different Laplace numbers, spanning over orders of magnitude with respect to density-ratio ($\rho_l/\rho_g$). We observe that the spurious currents do not decay to machine precision as in static droplet case, instead oscillate around a mean value of the order of $0.1 \% $ of the constant field $U$. For larger density-ratios, there is a slight upward trend with respect to a time-scale much larger than $T_U$, with the oscillations corresponding to a time-scale of the order $U/\Delta x$. All of the plots in figures \ref{evo_nonmc} tp \ref{evo_sagar} correspond to $We = 0.4$, alongside an additional simplification of no viscosity contrast across the interface i.e $\mu_l/\mu_g = 1$ .

As evindenced by the persistence of the spurious currents due to the addition of interfacial propagation, further advancements should be made with respect to the combined performace of the interfacial transport, curvature computation and the surface tension model. Nonetheless, the present method seems to be quite numerically stable when dealing with the high density-ratios, and are not subject to rapid uncontrollable amplifications of the interfacial perturbations even for high Laplace numbers.

\begin{figure}[h!]
    \centering
    \includegraphics[]{plots/droplet_advect/evo_nonmc.png}
	\caption{\textbf{M1} Decay of normalized spurious currents as a function of viscous dissipation time-scales for different density-ratios and Laplace numbers. The currents seem to initially decay quickly for all higher density-ratios, and relax to the numerical equilibrium curvature even within $0.2 \cdot T_\mu$. For combinations of large $\rho_l / \rho_g$ and large $La$, the spurious currents seem to grow back to an order of magnitude ($10^{-8}$) which is quite far from that of machine precision ($10^{-14}$).}   
    \label{evo_nonmc}
\end{figure}

\begin{figure}[h!]
    \centering
    \includegraphics[]{plots/droplet_advect/evo_daniel.png}
	\caption{\textbf{M2} Decay of normalized spurious currents as a function of viscous dissipation time-scales for different density-ratios and Laplace numbers. The currents seem to initially decay quickly for all higher density-ratios, and relax to the numerical equilibrium curvature even within $0.2 \cdot T_\mu$. For combinations of large $\rho_l / \rho_g$ and large $La$, the spurious currents seem to grow back to an order of magnitude ($10^{-8}$) which is quite far from that of machine precision ($10^{-14}$). No considerable improvement is observed with respect to \textbf{M1}. }   
    \label{evo_daniel}
\end{figure}

\begin{figure}[h!]
    \centering
    \includegraphics[]{plots/droplet_advect/evo_sagar.png}
	\caption{\textbf{M3} Decay of normalized spurious currents as a function of viscous dissipation time-scales for different density-ratios and Laplace numbers. The currents seem to decay very quickly in the case of higher density-ratios, and relax to the numerical equilibrium curvature even within $0.2 \cdot T_\mu$. For all combinations of $\rho_l / \rho_g$ and $La$ numbers, the decayed spurious currents are not observed to grow back as in the cases of \textbf{M1} and \textbf{M2}, and hover around values close to machine precision ($10^{-14}$).}   
    \label{evo_sagar}
\end{figure}

\subsection*{Spatial Convergence}

In fig. show the scaling of the error as a function of spatial resolution for the most stringent case of $\rho_l/\rho_g = 1000 $ , $La = 12000$. The error seems to convergence at somewhat less than first-order for the $L_\infty$ norm, and slightly faster than a first-order rate for the $L_2$ norm. This seems to be consistent with earlier studies (\cite{popinet2009accurate}) which were carried out for equal density fluids.


\subsection*{Error Dependence : Laplace \& Weber numbers}

Fig. demonstrates the behavior of the error as a function of differenet $We$ and $La$ numbers, all carried out for the largest density-ratio ($\rho_l/\rho_g$). The error (both $L_\infty$ and $L_2$) scales as $We^{-1/3}$, which is marginally different from the $We^{-1/2}$ scaling observed by Popinet \cite{popinet2009accurate} (although \cite{popinet2009accurate} had equal densities ($\rho_l/\rho_g = 1$)). In terms of Laplace numbers, the errors scale as $La^{1/6}$ , which is the same as that observed in \cite{popinet2009accurate} (again, for equal densities).


%---------------------------------------------------------------

\section{Capillary Wave}
One of fundamental features of immiscible multiphase flows involving interfaces are the presense and propagation of capillary waves. Therefore, a robust and accurate numerical method should not only be able to adequately resolve, but also accurately emulate the spatio-temporal evolution of such surface tension induced oscillations. A brief outline on the numerical implementations of capillary waves in existing literature is provided by Popinet in the comprehensive review \sidecite{popinet2018numerical}.

\subsection*{Setup}

\begin{figure}[h!]
    \centering
    \includegraphics[width = 0.6\textwidth]{plots/capwave/config.png}
	\caption{Schematic of the initially perturbed planar interface separating two immiscible fluids of different densities and viscosities. A spatial resolution of $32 \times 96$ is used for spatial discretization (compared to $64 \times 192$ in Popinet \cite{popinet2009accurate}), with the width of the box corresponding to the size of the perturbed wavelength.}
    \label{capwave_conf}
\end{figure}


Ideally, we would like to evaluate the accuracy of our method compared to the analytical solution of damped capillary oscillations. Analytical solutions exist for the case of extremely small initial perturbations, either in the inviscid limit (Lamb \sidecite{lamb1993hydrodynamics}) or the asymptotic limit of vanishing viscosity (Prosperetti \sidecite{prosperetti1980free,prosperetti1981motion}). In the present study, we use the configuration of the viscosity-damped capillary oscillations of a planar interface, as was first implemented and popularized by Popinet \& Zaleski \cite{popinet1999front}.  

We consider a rectangular domain of dimensions $L \times 3L$, where $L$ corresponds to the wavelength of our initial perturbation. The densities of the heavier and lighter phases are $\rho_l$ and $\rho_g$ respectively, likewise for the viscosities $\mu_l$ and $\mu_g$, and $\sigma$ being the surface tension coefficient (fig. \ref{capwave_conf}). An intial perturbation amplitude of $L/100$ is used, coupled with a numerical resolution given by $L/\Delta x= 32$ ($\Delta x$ being the grid size). Symmetry conditions are applied on the top and bottom sides, with periodic conditions along the horizontal direction. We use the following adimensional parameters to characterize our problem : 

\begin{align}
	T_0 = T \omega_0 \quad , \quad La = \frac{\rho_l \sigma L}{\mu_l^2}  
\end{align}

where $La$ is the Laplace number based on the heavier fluid, and $\omega_0$ is defined using the dispersion relation \cite{popinet2009accurate} given as : 

\begin{align}
	\omega_0^2 =  \frac{\sigma k^3}{2 \rho_l} \quad, \qquad \text{where} \quad k = \frac{2\pi}{L}   
\end{align}

The dispersion relation is obtained via linear stability analysis at the inviscid limit \cite{lamb1993hydrodynamics}. In order to evaluate the influence of density-ratio on the performance of our method, we use 3 different numerical setups keeping the same Laplace number ($La = 3000$) as follows : 

\begin{itemize}
	\item $\rho_l/\rho_g = 1$ , $\mu_l/\mu_g = 1$  (Popinet \cite{popinet2009accurate}) 
	\item $\rho_l/\rho_g = 10$ , $\mu_l/\mu_g = 1$   
	\item $\rho_l/\rho_g = 1000.0/1.2$ , $\mu_l/\mu_g = 1.003\cdot 10^{-3}/1.8\cdot 10^{-5}$ (Air-Water) 
\end{itemize}
% verify the laplace numbers in the adimensional form and paris scripts


\subsection*{Comparison with Prosperetti Solution}
\blindtext

\subsection*{Spatial Convergence}
\blindtext

%---------------------------------------------------------------

\section{Falling Raindrop}

A flow configuration that combines the complexities of high density-ratios with the interaction between capillary, viscous and inertial stresses is that of a water droplet falling in air under the influence of gravitational acceleration. 


\subsection*{Setup}
The problem is characterized by a combination of Reynolds, Weber and Bond numbers, the definitions of which are as follows : 

\begin{align}
We=\frac{\rho_{g} U^2 d}{\sigma} \quad,\quad Re= \frac{\rho_{g} U d}{\mu_{g}} \quad,\quad Bo=\frac{\left(\rho_{l}-\rho_{g}\right) g d^2 }{\sigma}
\end{align}

where the subscripts $l$ and $g$ refer to the liquid phase (water) and the gas phase (air), respectively. In our particular numerical setup, $We \simeq 3.2 $, $Re \simeq 1455 $ and $Bo \simeq 1.2 $, thus corresponding to that of a $3mm$ diameter raindrop (a relatively large one) falling in air at an approximate terminal velocity of  $8$ m/s (interpolated from empirical data, refer to  \cite{gunn1949terminal}). The parameters in the problem setup are given in Table \ref{raindropprop}, and the schematic diagram given by Fig. \ref{setup}. The droplet is initially placed at the center of a cubic domain (3D), whose side is 4 times the diameter of the drop. 

% Table of droplet setup parameters 
\begin{table*}[h!]
\begin{center}
\begin{tabular}{ccccccc}
\hline\hline
$\rho_{g}$ & $\rho_{l}$ & $\mu_{g}$ 
& $\mu_{l}$ & $\sigma$ & $d$ & $g$\\
$\left(kg/m^3\right)$ & $\left(kg/m^3\right)$ & $\left(Pa \, s\right)$ 
& $\left(Pa \,s \right)$ & $\left(N/m\right)$ & $(m)$ & $(m /s^{2})$ \\
\hline
1.2 & $0.9982 \times 10^3$ & $1.98 \times 10^{-5}$ & 
$8.9 \times 10^{-4}$ & $0.0728$ & $3 \times 10^{-3}$ & $9.81$\\
\hline\hline
\end{tabular}
\caption{Parameter values used in the simulation of a falling water droplet in air. 
\label{raindropprop}}
\end{center}
\end{table*}
% -----

In order to properly reproduce and analyse the dynamics of a relatively large drop (high Reynolds flow) such as in our case, the numerical method has to accurately resolve the thin boundary layers, the interaction of such layers with the capillary forces and finally the non-linear feedback of the complex 3D vortical structures present in the wake behind the droplet. Therefore, our objective behind the demonstration of this particular test case is not to develop a high fidelity model of a raindrop, but rather carry out a stringent evaluation of the robustness of our numerical method compared to the standard version of our solver which is not momentum consistent. For such a low Weber number the capillary forces dominate and the droplet should remain intact, and definitely not undergoe any subsequent atomization. 


% -----
% Schemetic of case setup with boundary conditions
\begin{figure}[h!]
\begin{center}
\includegraphics[width=0.8\textwidth]{plots/raindrop/setup.png}
\end{center}
\caption{A 2D schematic of the numerical setup for the falling raindrop. We apply a uniform inflow velocity condition with $U_0(t)$ and an outflow velocity condition at the top which corresponds to zero gradient of the velocity at the boundary. Boundary conditions on the side walls correspond to those of free slip (no shear stress).}
\label{setup}
\end{figure}


\subsection*{Temporal Evolution : Kinetic Energy, Mass, Moment of Inertia}
\blindtext

\subsection*{Convergence of Velocity \& Acceleration}
\blindtext


