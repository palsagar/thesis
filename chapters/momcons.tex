\setchapterpreamble[u]{\margintoc}
\chapter{Consistent Mass-Momentum Transport}
\labch{momcons}

% intro : PARIS Simulator, necessity due to large density-ratios ? 

The stability of numerical methods that attempt to model 
interfacial flows involving large density contrasts 
face several challenges, key amongst them being the transport of 
mathematical discontinuities that arise from the contrast in
the material properties of the fluids involved. 
Extremely small numerical errors are ubiquitous as a consequence of the 
numerous approximations involved at each and every step of the algorithm 
\sidenote{Approximations are made while interfacial propagation
, curvature computation, surface tension modeling etc.}. 
In the context of such flows, such errors generally result in physically
inconsistent mass and momentum transfer across the interface, often from
the denser phase towards the lighter phase.
The presence of large density or viscosity contrasts tend to amplify
the growth of these cascading numerical errors, eventually
leading to significant (often catastrophic) interfacial deformations
followed rapidly by a loss of numerical stability. 
In this chapter, we present a detailed exposition of our class 
of numerical methods which are designed specifically to circumvent or
suppress the growth of such numerical instabilities when dealing with
flows entailing large density-ratios. 
The implementations of the numerical algorithms are developed for 
finite volume discretizations using uniform Cartesian grids in 3D, 
on the `PARIS Simulator' \cite{paris} numerical platform.      
Some contents of the chapter are primarily derived 
from the recent study by Fuster et al. \sidecite{caf_momcons}, 
in which the author is a primary contributor. 

% ----------------------------------------------------------------------------------------------------------------------

\section{General Principles}

%intro : what do you mean by consistent transport? 
In the past two decades, considerable efforts have been made 
in the design of numerical methods to deal with large density-ratios. 
The underlying principle behind these endeavours is that the governing 
equations for the transport of mass and momentum are modeled using a 
conservative formulation (divergence of fluxes), instead of standard 
non-conservative forms which themselves were adapted directly from 
techniques developed originally for single-phase flows. 
This formulation enables one to render the discrete transport of 
momentum \textit{consistent} with respect to the discrete transport of mass  
\sidenote{First of all, the transport of mass (density) has to be made
consistent with that of the geometric transport of volume fraction.}. 
Such a tight coupling of the propagation of errors between the discrete 
mass and momentum fields enables alleviation of many of the issues 
that plague such numerical methods, especially in the 
context of low to moderate spatial resolutions. 
A secondary but important mitigating factor is the advancements 
made in the modeling of capillary forces, resulting in the adoption of 
consistent and \textit{well-balanced} surface tension formulations.
\sidenote{Consistency in the context of surface tension models refers to the ability 
of methods to progressively achieve more accurate estimations of interfacial 
curvature as a result of increasing spatial resolution, 
whereas well-balanced refers to the ability to recover certain static 
equilibrium solutions pertaining to surface tension dominated flows 
without the perpetual presence of parasitic or spurious currents in the velocity fields.}.
We refer the reader to influential works of Popinet \cite{popinet2018numerical,popinet2009accurate}
to get a better understanding of the issues surrounding surface tension implementations.  

\subsection*{Major Iterations in Literature}

% literature review 

% Rudman (1998) :
The first study to address the issue of consistency between mass 
and momentum transport was the seminal work of Rudman \sidecite{rudman1998volume}.
The fundamental hurdle in the implementation of mass-momentum 
consistent transport for staggered configurations of primary variables 
(pressure and velocity) is the inherent difficulty in reconstructing mass 
(defined on centered control volumes) and its corresponding fluxes 
onto the staggered control volumes on which momentum is defined. 
Rudman introduced the strategy of carrying out mass advection 
\sidenote{Mass advection was carried out using using algebraic flux reconstructions.} 
on a grid twice as fine as that of momentum, 
thereby enabling a `natural' and intutive 
way to reconstruct mass and its fluxes onto staggered momentum control volumes. 
However, the method uses a VOF based convolution technique for curvature computation, 
which is neither consistent nor well-balanced.    


% Bussmann et al. (2002) :
Bussmann et al. \sidecite{bussmann2002modeling} were able to 
circumvent the issue surrounding staggered grids altogether 
by using a collocated arrangement in the context of hexahedral 
unstructured meshes, coupled with an unsplit Eulerian flux computation method.    
The study though makes no mention of any surface tension model. 


% Raessi & Pitsch (2012) | Ghods & Hermann (2013):
Level set based methods in the context of mass-momentum consistent 
transport were implemented first by Raessi and Pitsch \sidecite{raessi2012consistent}, 
followed by Ghods and Hermann \cite{ghods2013consistent}. 
In the former, the consistency problem is tackled by means of a 
semi-Lagrangian approach, computing geometric level set derived 
fluxes at two different time intervals, whereas in the latter, 
a collocated arrangement is used. 
Nonetheless, both methods face certain drawbacks, notably the 
applicability only to 2D in case of Raessi and Pitsch, 
as well as a lack of well-balanced surface tension models for both these methods.   


% LeChenadec & Pitsch (2013) | Owkes & Desjardins (2017) : 
Recent advances concerning volume-of-fluid based methods that 
employ unsplit (conservative) geometric flux reconstructions 
were made by LeChenadec and Pitsch \sidecite{le2013monotonicity}
, and later by Owkes and Desjardins \cite{owkes2017mass}. 
LeChenadec and Pitsch utilize a Lagrangian remap method in order 
to construct consistent mass-momentum fluxes for the staggered control volumes, 
while Owkes and Desjardins use mass advection on a doubly refined grid 
(same principle as Rudman) to achieve consistency.     
Although \cite{le2013monotonicity} implements a well-balanced surface tension model, 
the VOF convolution based curvature computation is not consistent. 
In case of \sidecite{owkes2017mass}, they use mesh-decoupled height functions 
to compute curvature while coupling it with a well-balanced surface tension model. 
However, their semi-Lagrangian flux computation procedure 
involving streak tubes and flux polyhedra are extremely convoluted in 3D.     


% Vaudor et al. (2017) | Zuzio et al. (2020) : 
Certain methods attempt to combine the qualities of both 
VOF and level set methodologies (CLSVOF), as proposed in the works of 
Vaudor et al. \sidecite{vaudor2017consistent}, 
and more recently by Zuzio et al. \cite{zuzio2020new}. 
They both tackle the consistency issue by means of projecting 
the direction-split geometric fluxes onto a twice finer grid, 
which are subsequently recombined to reconstruct consistent 
fluxes for mass and momentum for the staggered control volumes. 
This approach allows them to bypass the requirement of conducting 
mass advection on a twice finer grid (as in the original Rudman method), 
thereby deriving the benefits of a sub grid without the 
added computational cost of doubly refined mass transport. 
In addition, both Vaudor et al. \cite{vaudor2017consistent} and 
Zuzio et al. \sidecite{zuzio2020new} adopt well-balanced surface tension 
models with consistent level set based curvature estimation. 
However, the purported advantages of both these methods with 
regards to reduced computational costs is not quite evident, 
as additional complexities are introduced due to the projection 
(reconstruction) of fluxes onto a the twice finer mesh, 
which would not be necessary in the first place if mass transport 
had been carried out on the twice finer mesh itself. 


% Patel & Natarajan (2017) : 
Patel and Natarajan \sidecite{patel2017novel} developed a hybrid 
staggered-collocated approach to solve the consistency issue on 
polygonal unstructured meshes, complemented with a well-balanced surface tension model. 
Nevertheless, the VOF advection is based on algebraic transport, 
not to mention the use of a VOF convolution based 
curvature computation, which is inherently not consistent. 


% Nangia et el. (2019) : 
More recently, Nangia et al. \sidecite{nangia2019robust} developed a 
CSLVOF method for dynamically refined staggered Cartesian grids. 
They utilize Cubic Upwind Interpolation (CUI) schemes to 
reconstruct consistent mass and momentum fluxes on the staggered 
control volumes, using the information from the additional mass 
advection equation they solve alongside the level set function.   
However, the reconstruction of mass fluxes using CUI schemes are 
inherently algebraic, with their comparative advantage against fluxes 
computed via geometric constructions being an open question 
\sidenote{refer to Mirjalili et al. \cite{mirjalili2017interface}. }.

To get a bird's eye view of the numerous features employed by the methods 
in existing literature, we refer the reader to tables %\ref{table_vof} and \ref{table_ls}, 
which respectively provide systematic overviews of the VOF based and level set based approaches.    

% modified comparison charts with CAF paper added



\subsection*{Our Approach}

In the present body of work, we start by precisely defining the
essential (desired) attributes of a numerical scheme that 
ensures discrete consistency between mass and momentum transport : 

\begin{itemize}
	\item The discontinuity of the mass (volume fraction 
		derived density field) should propagate at the 
		exact `numerical' speed as that of 
		the discontinuity of the momentum field. 
	\item The numerical transport of momentum should be performed 
		in a manner consistent to the transport of mass for 
		each direction, implying that the momentum fluxes must 
		be obtained directly from geometrically 
		computed fluxes of mass (volume). 
\end{itemize}

In order to tackle the challenge of consistent transport on staggered control 
volume configurations, we have developed two different strategies, namely, 
the \textit{shifted fractions} method and the \textit{sub-grid} method. 
The former uses geometrical reconstructions to derive a 
\textit{shifted} volume fraction field which is centered on the staggered
control volumes, whereas the latter adopts the Rudman \cite{rudman1998volume} 
strategy of volume advection on a twice finer grid in order to 
enable consistent reconstruction of mass and momentum on the staggered control volumes. 
Another key contribution of this body of work is to extend the conservative 
direction-split mass transport algorithm of Weymouth and Yue \sidecite{wy} 
to the direction-split transport of momentum. 

A thorough description of the shifted fractions method is can be found in
the recent study by Fuster et al. \sidecite{caf_momcons} 
, and a detailed exposition of the
sub-grid method is currently under preparation at the time of writing.  
We shall be taking a closer look at these two stratetgies in the upcoming sections.  


% ----------------------------------------------------------------------------------------------------------------------


\section{Basic Expressions}

% intro : interfacial propagation 

\newcommand{\dert}[1]{\frac{\partial #1}{\partial t}}
\newcommand{\cijk}{C_{i,j,k}}
\newcommand{\pijk}{\phi_{i,j,k}}
\newcommand{\X}{\boldsymbol{x}}
\newcommand{\N}{\boldsymbol{n}}
\newcommand{\rijk}{\rho_{i,j,k}}
\newcommand\mijk{(\rho u_{q})_{i,j,k}}

We start with the advection of the interface position, which in 
the VOF framework is given by - 

\marginnote{In the VOF approach, the approximation to the interface Heaviside 
function contains a sharp discontinuity, in contrast to the level set approach 
where the approximation is continuous in the discrete sense.}


\begin{align}
	\dert H + \nabla \cdot (\boldsymbol{u} H) = H\, \left(\nabla \cdot \boldsymbol{u} \right) \,
	\label{heaviside}
\end{align}


Integrating this equation in time after carrying 
out spatial discretization, one obtains -    


\marginnote{$\cijk^n$ is the colour function or volume fraction field at time step $n$ 
obtained through the finite volume discretization of the interface Heaviside function.}


\begin{align}
{\cijk^{n+1} - \cijk^{n}} = - \sum_{\textrm{faces} \, f} F^{(c)}_f + \int_{t_n}^{t_{n+1}}  
	{\textrm{d}}t \int_\Omega  H \, \left(\nabla \cdot \boldsymbol{u}\right) \,  {\textrm {d}}\X   \,,
\label{sumf}
\end{align}


where the first term on the right-hand side is the summation over the cell faces $f$  
of the fluxes $F^{(c)}_f$   
\sidenote{These fluxes of $(\boldsymbol{u} H)$ are computed via geometrical reconstructions,
and not via high-order non-linear interpolation schemes which are 
the standard (in the absence of discontinuities) . } . 
As one can clearly observe, the ``compression'' term on the 
right-hand side of equation \eqref{heaviside} disappears for incompressible flow, 
however it is essential in the context of direction-split geometric advection schemes.
The definition of the geometric fluxes in mathematical form is -  

\begin{align}
	F_f^{(c)} = \int_{t_n}^{t_{n+1}} {\textrm {d}}t \int_{f} u_f(\X,t) H(\X,t) \, {\textrm{d}}\X \,,
\label{faceint}
\end{align}

where $u_f = \boldsymbol{u}\cdot \N_f$ 
\sidenote{Component of the velocity normal to the control surface $f$.}.
Once an approximation for the evolution of $(\boldsymbol{u} H)$ during
the time step is chosen, a four-dimensional integral remains to be
computed in equation (\ref{faceint}).
The advection schemes used are CIAM and Weymouth-Yue, which have been 
described briefly in chapter \ref{ch:method}. 

Directional splitting results in the breakdown of equation (\ref{sumf}) 
into three equations, one for each advection substep -


\begin{align}
{\cijk^{n,l+1} - \cijk^{n,l}} = - F^{(c)}_{m-} - F^{(c)}_{m+} 
+ c_m \partial_{m}^h u_m \,,
\label{sumf2}
\end{align}


\marginnote{The superscript $l=0,1,2$ is the substep index, i.e.
$\cijk^{n,0} = \cijk^{n}$ and $\cijk^{n,3} = \cijk^{n+1}$. 
The face with subscript $m-$ is the ``left'' face in direction $m$ with 
$F^{(c)}_{m-} \ge 0$ if the flow is locally from right to left. A similar reasoning 
applies to the ``right'' face $m+$.}

After each advection substep (\ref{sumf2}), the interface is reconstructed
with the updated volumes $\cijk^{n,l+1}$, then the 
fluxes $F^{(c)}_{f}$ are computed for the next substep. 
Importantly, we have approximated the compression term in (\ref{sumf}) by -

\begin{align}
	\int_{t_n}^{t_{n+1}}  {\textrm {d}}t \int_\Omega  H \partial_m u_m  {\textrm {d}}\X \simeq  c_m 
 \partial_{m}^h u_m \, 
 \label{comp}
\end{align}

\marginnote{There is no implicit summation carried out 
over $m$, and the superscript $h$
denotes the spatial discretization of the operator. } 

On the right hand sides of (\ref{sumf2}) and (\ref{comp}) 
the flux terms $F_{f}^{(c)}$ and the 
 partial derivative $\partial_{m} u_m$ must
be evaluated using identical discretized velocities.
The expression $\partial_{m}^h u_m$ is a finite volume approximation of the 
spatial derivative corresponding to the $m$th component 
of the velocity vector along direction $m$, and
the ``compression coefficient'' $c_m$ approximates the color fraction. 
Its exact expression is dependent on the advection method
\sidenote{Refer to chapter \ref{ch:method} to see the difference in
the definitions of the compression coefficient between  
the CIAM and Weymouth-Yue methods.}
and it also entails the desirable property of $C$-bracketing
\sidenote{The preservation of $0 \le \cijk \le 1$ 
is referred to as $C$-bracketing. }. 
Due to the possible dependence of the 
compression coefficient $c_m$ on the Cartesian direction
corresponding to the advection substep, 
the sum $\sum_m c_m \partial_{m}^h u_m$ may not necessarily vanish
even if the flow is incompressible.
As mentioned previously in chapter \ref{ch:method}, the main appeal
of the Weymouth-Yue method is its ability to keep this sum to zero,
within the limits of machine precision. 

\subsection*{Advection of Conserved Scalars}  

In order to achieve consistent transport of the discontinuous fields 
of mass and momentum, we start by trying to understand the advection 
of a generic \textit{conserved} scalar quantity $\phi$ by a continuous velocity field - 


\begin{align}
	\dert \phi + \nabla \cdot \left( \phi \, \boldsymbol{u} \right)  = 0 
\label{phiconv}
\end{align}


where the field $\phi$ is smoothly varying except 
at the interface position, where it may be discontinuous.
The essence of this study lies in the search of a scheme that  
propagates this discontinuity ($\phi$) at the same speed as 
that of the advection of volume fraction ($C$).

\marginnote{The smoothness of the advected quantity away from the interface
is verified for fields such as density $\rho$, momentum $\rho \boldsymbol{u}$ 
and internal energy $\rho e$.}

Temporal integration of the spatially discretized version of \eqref{phiconv} gives us - 

\begin{align}
\pijk^{n+1} - \pijk^{n} = - \sum_{\textrm{faces}\, f} F^{(\phi)}_f 
\label{sumfp}
\end{align}

\marginnote{The sum on the right-hand side is the sum over faces $f$ of cell $i,j,k$ 
of the fluxes $F^{(\phi)}_f$ of $\phi$, which are defined as the color function fluxes
$F^{(c)}_f$ in \eqref{faceint}.}


where the fluxes are defined as -

\begin{align}
	F_f^{(\phi)} = \int_{t_n}^{t_{n+1}} {\textrm {d}}t \int_{f} u_f(\X,t) \,\phi(\X,t) \,
	{\textrm {d}}\X
\label{pfaceint}
\end{align}

In order to ``extract'' the discontinuity we introduce the 
 interface Heaviside function $ H(\X,t) $

\begin{align}
F_f^{(\phi)} = 
\int_{t_n}^{t_{n+1}} {\textrm d}t \int_{f} \left[ u_f  \,H \,\phi  +  u_f \,(1-H) \,\phi \right] 
\, {\textrm d}\X \, 
\label{fluxphi}
\end{align}

Therefore, the flux can be decomposed into two components as -

\begin{align}
F_f^{(\phi)} = 
\bar \phi_1 \int_{t_n}^{t_{n+1}} {\textrm d}t \int_{f} u_f \,H \,{\textrm d}\X + 
\bar \phi_2 \int_{t_n}^{t_{n+1}} {\textrm d}t \int_{f} u_f \,(1-H) \,{\textrm d}\X \,,
\label{barphi}
\end{align}

where the face averages $\bar \phi_s$, $s=1,2$, are defined as - 


\begin{align} 
\bar \phi_s = \frac{\int_{t_n}^{t_{n+1}} {\textrm d}t \int_{f} \phi \,u_f \,H_s  
{\textrm d}\X}{\int_{t_n}^{t_{n+1}} {\textrm d}t \int_{f}  
u_f \,H_s\,{\textrm d}\X} \,,  
\label{barphi2}
\end{align}


and $H_1=H$, $H_2= 1-H$. 
The total flux can be rearranged as a sum of the constituents 
corresponding to the different `fluids' - 

\marginnote{Expression \eqref{barphi} can be written in terms 
of the fluxes $F^{(c)}_f$ and $F^{(1-c)}_f$, this second one being obtained by 
replacing $H$ with $1-H$ in \eqref{faceint}}

\begin{align}
F_f^{(\phi)} = \bar \phi_1 \,F_f^{(c)} +  \bar \phi_2 \,F_f^{(1-c)} \,.
\label{fluxphi12}
\end{align}



\subsection*{Advection of Density}
% VOF - rho consistency with compression term etc

The density field $\rho(\X,t)$ follows the 
temporal evolution of the generic conserved quantity
(\ref{phiconv}) by setting $\phi = \rho$. 
At the incompressible limit the velocity field is 
solenoidal (divergence-free), with constant densities 
in each phase. We can extract the density trivially from the integrals 
\eqref{barphi2} to \textit{exactly} obtain $\bar \rho_s = \rho_s$. 
The flux definitions corresponding to $\rho$ becomes -   


\begin{align}
F_f^{(\rho)} = \rho_1 F_f^{(c)} +  \rho_2 F_f^{(1-c)}
\label{fluxrho}
\end{align}

\marginnote{We use the term `conservative' to express the fact
that eq. (\ref{sumfp}) computes the temporal evolution of $\rho$ 
as a difference of fluxes, instead of velocity times its gradient.}


Using the above definitions for density fluxes,
one can employ any VOF method to construct fluxes of the color 
function in order to obtain conservative transport for $\rho$
In principle, this should result in the conservation of total mass.  
However, as we have already pointed out the fact that the CIAM
method does not conserve volume exactly 
\sidenote{This again comes back to the dependence of the compression 
coefficient on the volume fraction field at the start of each
advection substep.}
, consequently, the advection of the density field (mass) 
is not consistent with that of the advection of $C$. 

This apparent paradox is resolved by including the 
compression term on the right hand side of \eqref{phiconv},
in order to ensure consistency. 
The advection equation for the conserved quantity now becomes - 


\begin{align}
\dert \phi + \nabla \cdot ( \phi \,\boldsymbol{u})  = \phi \, \left(\nabla \cdot \boldsymbol{u}\right) 
\label{phiconv2}
\end{align}


This equation can be decomposed into advection substeps corresponding to 
the direction-split integration framework as follows - 

\marginnote{The direction-split integration steps operations for $\phi$
mirror that of the volume fraction $C$. }


\begin{align}
{\pijk^{n,l+1} - \pijk^{n,l}} = - F^{(\phi)}_{m-} - F^{(\phi)}_{m+} 
+ \Big( \tilde \phi_1^m c^{(1)}_m + \tilde \phi_2^m c^{(2)}_m \Big) \, \partial_{m}^h u_m 
\label{sumfpconsistent}
\end{align}


\marginnote{$c^{(1)}_m=c_m$ is the compression coefficient corresponding 
to the particular VOF advection method for volume fraction $C$ ,
while $c^{(2)}_m= 1 -c_m$ corresponds to that of the 
symmetric color fraction  $1 - C$.}


The fluxes $F^{(\phi)}_{m\pm}$ follow the definition given in \eqref{fluxphi12}, 
, while the cell averages $\tilde \phi_s^m$ are defined as - 


\begin{align}
\tilde \phi_s^m = \frac{\int_{t_n}^{t_{n+1}} {\textrm{d}}t \int_{\Omega}  \phi  H_s  
\partial_{m}^h u_m  \,  {\textrm{d}}\X}
{\int_{t_n}^{t_{n+1}} {\textrm{d}}t \int_{\Omega} H_s  \partial_{m}^h u_m \,{\textrm{d}}\X} \,
\label{cellphi2}
\end{align}


Rewriting the direction-split integration operation for $\rho$ we get - 


\marginnote{In case of CIAM, the compression coefficient is defined as
$c_m = \cijk^{n,l}$, whereas in Weymouth-Yue the coefficient is independent
of the advection substep $l$, defined as  
$c_m = H\left(\cijk^{n} - 1/2 \right)$, where $H$ is the Heaviside function.}


\begin{align}
{\rijk^{n,l+1} - \rijk^{n,l}} = - F^{(\rho)}_{m-} - F^{(\rho)}_{m+} + C_m^{(\rho)}\,,
\label{sumfrho}
\end{align}


where the fluxes are given by (\ref{fluxrho}) and the compression term is

\marginnote{Again, there is no implicit summation rule on $m$.} 


\begin{align}
C_m^{(\rho)} =  \left( \rho_1 c^{(1)}_m + \rho_2 c^{(2)}_m \right) \,\partial_{m}^h u_m  
\label{central}
\end{align}



For the WY method, the compression terms eventually cancel upon 
summation over the substeps, therefore resulting in the conservation  
of mass at the same accuracy as the discrete incompressibility condition
$\sum_{m=1}^3 \partial_{m}^h u_m=0$ is verified 
\sidenote{This usually corresponds to the tolerance of the Poisson solver,
with a typical value being $10^{-6}$. }. 



\subsection*{Advection of Momentum}
% model of computing consistent mass-momentum fluxes

Within the framework of advection of a generic conserved scalar,
we consider momentum advection as transport of the scalar quantities  
$\phi=\rho u_q$, where $q=1,2,3$ is the component index. 
Using the definition given in \eqref{barphi2}, 
we obtain the expression -  

\marginnote{The expressions with the `bar' 
refer to face weighted averages of the varible in question. 
Here, $\bar \phi_s = \overline { \rho u_q}_s$ }


\begin{align} 
\overline{\rho u_q}_s = \rho_s \bar u_{q,s} 
\end{align}


where $\bar u_{q,s}$ is termed as the ``advected interpolated velocity'', 
whose precise definition is given as - 


\begin{align}
	\bar u_{q,s} =  \frac{\int_{t_n}^{t_{n+1}} {\textrm{d}}t \int_{f}  u_q u_f  H_s  \, {\textrm{d}}\X}{\int_{t_n}^{t_{n+1}} {\textrm{d}}t \int_{f}  u_f  H_s\,{\textrm{d}}\X} \label{barudef}
\end{align}

\marginnote{The subscript $s$ denotes the phase or fluid,
and $f$ represents the normal components defined on the face
centers.} 


Thus, the direction-spit integration of the momentum can be represented as - 


\begin{align}
{\mijk^{n,l+1} - \mijk^{n,l}}  = - F^{(\rho u)}_{m-} - F^{(\rho u)}_{m+}
 + \Big( \rho_1 \tilde u_{q,1}^m  c^{(1)}_m +  \rho_2 \tilde u_{q,2}^m c^{(2)}_m 
 \Big) \, \partial_{m}^h u_m 
\label{sumfrou}
\end{align}


where the momentum fluxes are constructed in the following manner -


\begin{align}
 F^{(\rho u)}_{f} =  \rho_1 \,\bar u_{q,1}  \,F^{(c)}_{f}  +  
 \rho_2 \,\bar u_{q,2}  \,F^{(1-c)}_{f} \,,
\end{align}


The expression for the ``central interpolated velocity'' corresponding to the 
averages $\tilde \phi_s^m$ of \eqref{cellphi2} is -  

\marginnote{Notice that ``cloning'' the advected velocities 
$\bar u_{q,1}$ and $\bar u_{q,2}$ 
would make it easier to advect a velocity field 
with a jump on the interface, but would 
render the overall algorithm quite complicated. }


\begin{align}
	\tilde u_{q,s}^{m} = \frac{\int_{t_n}^{t_{n+1}} {\textrm{d}}t \int_{\Omega} u_q \, H_s \,  
	\partial_{m}^h u_m   \,  {\textrm{d}}\X}
	{\int_{t_n}^{t_{n+1}} {\textrm{d}}t \int_{\Omega} H_s \, \partial_{m}^h u_m \,{\textrm{d}}\X} 
\label{tildeudef}
\end{align}


The superscript $m$ is intentionally omitted  
for the velocities $\tilde u_q^m$ in order to avoid
cumbersome and complicated notations. 

\marginnote{Due to the viscous effects and 
the absence of phase change in our fluid dynamics model,   
the velocity filed maintains continuity across the interface.} 

As a reasonable approximation we choose to put 
$\bar u_q =  \bar u_{q,1} = \bar u_{q,2}$ for the 
``advected interpolated velocity'' and 
$\tilde u_q =  \tilde u_{q,1} = \tilde u_{q,2}$
for the ``central interpolated  velocity''. 
An important simplification which serves as  
the central model in our development is given by -  

\marginnote{To add clarity to the notion of ``central interpolated velocity'', 
one can understand this as the face-centered interpolations of the velocity field 
(component normal to control surface), which is required to compute the fluxes of volume. }


\begin{align}
 F^{(\rho u)}_{f} = \bar u_q F^{(\rho)}_{f} \label{frou}
\end{align}


Therefore, the advection substep for the momentum can finally we written as -   


\begin{align}
{\mijk^{n,l+1} - \mijk^{n,l}} =  -\bar u_q  F^{(\rho)}_{m-} - \bar u_q  F^{(\rho)}_{m+} 
+ \tilde u_q C_m^{(\rho)}
\label{sumfmom2}
\end{align}


where the density fluxes are defined in \eqref{fluxrho} and the compression term 
$C^{(\rho)}$ in \eqref{central}. 


\marginnote{ In the above expression the face-weighted 
average velocities $\bar u_q$ are defined
using (\ref{barudef}) on the corresponding
left face $m-$ or right face $m+$. 
}

Although the weighted averages $\bar u_q$ and $\tilde u_q$ have been defined,
but the manner in which they are estimated shall be covered in the following section. 

In the context of the CIAM advection scheme, the compression coefficient
for the volume fraction field is $C^{n,l}$, similarly, for the 
central interpolated velocity we take $\tilde u_q = u_q^{n,l}$. 
Due to the summation of the directional divergences not cancelling out,
the resulting momentum transport after three directionally-split advections
and the result is not exactly conservative. 
On the other hand, the compression coefficient is independent 
of the advection substep in the context of the Weymouth-Yue advection scheme. 
The final expression for the compression coefficient becomes - 

\marginnote{The Weymouth-Yue coefficient is a constant value $c$,
that is independent of direction $m$.}

\begin{align}
C_m^{(\rho)} =  \Big( \rho_1 c + \rho_2 (1-c ) \Big) \,\partial_{m}^h u_m  \,.
\label{central2}
\end{align}


Since there is no bracketing on the any velocity component, 
we take $\tilde u_q = u_q^{n}$, which is independent of the substep $l$.
Provided that the velocity field is discretely solenoidal, i.e.  
$\sum_{m=1}^3 \partial_{m}^h u_m =0$, after the three direction 
split advections (\ref{sumfmom2}) one obtains 
a cancellation of the compression terms, finally resulting in 


\begin{align}
	{\mijk^{n,3} - \mijk^{n}} =  - \sum_{\textrm{faces} \, \textrm{f}}  \bar u_{q}  F^{(\rho)}_{f} . 
\label{sumfmomtotfracstep}
\end{align}


The extension of the Weymouth-Yue mass (volume) advection scheme to the consistent
transport of momentum allows the discrete transport to be exactly conservative. 



\subsection*{Velocity Interpolations}

% computation of slope limiters with diagram, distinction in bulk and interfacial regions

The momentum equation (\ref{sumfmom2}) can be approximated either
 1) in the bulk of the phases or 2) in the neighborhood of the interface.
In the first case the expression simplifies considerably since both the density
and the color fraction are constant and the spurious compression terms cancel out


\begin{align}
{u^{n,3}_q - u_q^{n}} =  - \sum_{\textrm{faces} \, \textrm{f}}  \bar u_{q} u_f 
\label{sumfmomtot}
\end{align}


We distinguish an ``advecting'' velocity $u_{f}= \boldsymbol{u} \cdot \N_f $ and an ``advected velocity'' 
component $\bar u_{q,f}$, involving an average over face $f$. Both velocity components
require an interpolation from their position in the staggered grid to where they are needed.
Thus the scheme in the bulk is 


\begin{align}
{u^{n,3}_q - u_q^{n}} =  - \sum_{\textrm{faces} \, \textrm{f}}  \bar u_{q}^{\textrm{(advected)}} 
u^{\textrm{(advecting)}}_f   
\label{sumfmomtotbulk}
\end{align}
	
while near the interface is 

\begin{align}
	{(\rho u_q)^{n,3} - (\rho u_q)^{n}} =  - \sum_{\textrm{faces} \, \textrm{f}}  \bar u_{q}^{\textrm{(advected)}}  
F^{(\rho)}_{f} + \sum_{m=1}^3 \tilde u_q C_m^{(\rho)} \, 
\label{advect-ed-ing-2}
\end{align}


Momentum advection in our model is described by these two equations which are
solved on a cubic grid with a finite-volume method. In the previous sections
we have derived a new expression for the momentum fluxes and the
compression term, i.e. the RHS of \eqref{advect-ed-ing-2}, that is consistent
with the volume fraction advection.  

To estimate the advecting velocities $u_f^{\textrm{(advecting)}}$ we use a centered scheme.
The staggered 2D grid of Fig. \ref{stag-grid} has the same variables
arrangement that is found in 3D on a plane perpendicular to the $z$-axis and through the 
pressure point $p_{i,j,k}$. To illustrate the procedure we consider a face perpendicular
to the horizontal direction $1$, in particular $f=1-$. 
There are two cases. In the first case the advected component is not aligned with the 
face normal, this corresponds to $q=2$. The $u_2$ control volume in 
Fig. \ref{stag-grid} is centered on $i,j+1/2,k$, and face $f=1-$ is then 
centered on $i-1/2,j+1/2,k$. The advecting velocity  $u_{1-}^{\textrm{(advecting)}}$ 
is not given on this point and has to be interpolated 


\begin{align}
 u_{1;i-1/2,j+1/2,k}^{\textrm{(advecting)}} = \frac12 \big( u_{1;i-1/2,j,k} + u_{1;i-1/2,j+1,k} 
\big) \,
\label{uf2}
\end{align}


In the second case the advected component is aligned with the 
face normal, this corresponds to $q=1$. The $u_1$ control volume 
in Fig. \ref{stag-grid} is centered on $i+1/2,j,k$ and face $f=1-$ is then 
centered on $i,j,k$. The interpolation is now


\begin{align}
	u_{1;i,j,k}^{\textrm{(advecting)}} = \frac12 \big( u_{1;i-1/2,j,k} + u_{1;i+1/2,j,k} \big) \,
\label{uf1} 
\end{align}
 
Now we turn to the interpolation of the \textit{advected} velocity $\bar u_{q}$ in
(\ref{advect-ed-ing-2}). The interpolants we use in this case are
one-dimensional and operate on the velocities $u_q$, on the center of their 
control volume, that are regularly spaced on a segment aligned 
with the direction of the advection, that is perpendicular to face $f$. 
We still consider an advection along the horizontal direction $1$. 
In Fig. \ref{advect-ed-ing-fig}, with the lighter notation $\phi = u_q$,
we need to interpolate the advected velocity on the left face
of the reference control volume $\Omega$.
For the advected velocity $u_1$ and the advecting velocity \eqref{uf1} on face
$f=1-$ the correspondence with the $\phi$ values in Fig. \ref{advect-ed-ing-fig} is


\begin{align}
\phi_{-3/2} = u_{1;i-3/2,j,k}, \quad \phi_{-1/2} = u_{1;i-1/2,j,k}, \quad   
\phi_{1/2} = u_{1;i+1/2,j,k}, \;\cdots
\end{align}


while for the advected velocity  $u_2$ and the advecting velocity \eqref{uf2} is


\begin{align}
\phi_{-3/2} = u_{2;i-2,j+1/2,k}, \quad  \phi_{-1/2} = u_{2;i-1,j+1/2,k}, 
\quad  \phi_{1/2} = u_{2;i,j+1/2,k}, \;\cdots
\end{align}


The extension of these results to an advection along the other two directions $q=2,3$ 
follows easily. We need to predict $\phi_0$ on face $f=1-$ in Fig. \ref{advect-ed-ing-fig}
to serve as an approximation of $\bar u_q$ given in (\ref{barudef}). We consider 
an interpolation function $f$ that computes 
this value as a function of the four nearest points,
and in an upwind manner based on the sign of the {\em advecting} velocity $u_f$


\begin{align}
	\phi_0 = f \big( \phi_{-3/2}, \phi_{-1/2}, \phi_{1/2},\phi_{3/2},{\textrm{sign}}(u_f)
\big) \,
\label{simpleinterp}
\end{align}

In this study we have extensively tested two kinds of interpolations:


\begin{enumerate}
\item a scheme that uses a QUICK third order interpolant in the bulk, away from the interface 
and a simple first order upwind flux near the interface. We call this scheme QUICK-UW;
\item a scheme that uses a Superbee slope limiter \cite{roe1985some} for the flux in the 
bulk and a more complex Superbee limiter tuned to a shifted interpolation point near the 
interface.  We naturally call this scheme ``Superbee''. 
\end{enumerate}


The details of the two schemes are given in \ref{appinterp}.


\begin{figure}
\begin{center}
    \includegraphics[width=\textwidth]{plots/advect-ed-ing.pdf}
\end{center}
\caption{The reference control volume $\Omega$ for the advected velocity component 
$\phi=u_q$ is shown. A horizontal advection is here considered and both the advecting velocity 
$u_f$ and the advected velocity require an interpolation for their value on the left face 
$f=1-$: (a) the value $\bar u_q = \phi_0$ (full circle) is interpolated 
(see \ref{appinterp}) from the values $\phi=u_q$ on the nodes (open circles);
(b) a more sophisticated interpolation predicts the value $\phi(\hat x)$ where $\hat x$ is
at the center of the ``donating'' region $\Omega_D$ (see \ref{appinterp}).}
\label{advect-ed-ing-fig}
\end{figure}






% ----------------------------------------------------------------------------------------------------------------------


\section{Advection on Staggered Grids}

% intro : staggered configuration of primary variables, control volumes for staggered advection 

\subsection*{Shifted Fractions Method}

% algorithm, flow chart , how to reconstruct on shifted grid, lack of conservation 

\subsection*{Sub-Grid Method}

% algorithm, flow chart, prolongation and restriction operations, consistent + conservative 

\subsection*{Source Terms}

% 1/rho terms in ST , reconstructing curvatures from sub-grid information etc 
