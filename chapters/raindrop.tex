\setchapterpreamble[u]{\margintoc}
\chapter{Artificial Atomization : The Falling Raindrop}
\labch{raindrop}

The focus of the present chapter shall be to take a closer look at the issues 
plaguing numerical methods that try to deal with flows 
entailing significant contrasts between the material 
properties of the two fluids. Arguably, the most common
instances of such flows are those involving the interaction
between air and water, primarily in the form of droplets and 
bubbles that play important roles in both natural and industrial processes. 
As for the numerical platform, we use `PARIS Simulator' 
\sidenote{Refer to \cite{paris} for a detailed exposition  
of the numerical methods implemented in `PARIS Simulator.} 
to carry out our simulations on this topic. 
A flow configuration that combines the complexities of large 
density-ratios with the interaction between capillary, viscous and 
inertial stresses is that of a water droplet falling in the 
air under the influence of gravitational acceleration.
The contents of the chapter are primarily derived 
from the recent investigations of the falling raindrop  
in Fuster et al. \sidecite{caf_momcons}, in which
the author is a primary contributor. 
%---------------------------------------------------------------
\section{Problem Setup}

The problem is characterized by a combination of Reynolds, 
Weber and Bond numbers, the definitions of which are as follows : 

\begin{align}
We=\frac{\rho_{g} U^2 D}{\sigma} \quad,\quad Re= \frac{\rho_{l} U D}{\mu_{g}} \quad,\quad Bo=\frac{\left(\rho_{l}-\rho_{g}\right) g D^2 }{\sigma}
\end{align}

The subscripts $l$ and $g$ represent liquid and gas phases respectively. 
In our particular numerical setup, $We \simeq 3.2 $, $Re \simeq 1455 $ and $Bo \simeq 1.2 $, 
thus corresponding to that of a $3mm$ diameter raindrop (a relatively large one) 
falling in the air at an approximate terminal velocity of  
$8$ m/s (interpolated from empirical data, refer to Gunn and Kinzer \cite{gunn1949}). 
The parameters in the problem setup are given in Table \ref{raindropprop}, 
and the schematic diagram given in Fig. \ref{setup}. 
The droplet is initially placed at the center of a cubic domain (3D), 
whose side is 4 times the diameter of the drop. 

\begin{table*}[h!]
\begin{center}
\begin{tabular}{ccccccc}
\hline\hline
$\rho_{g}$ & $\rho_{l}$ & $\mu_{g}$ 
& $\mu_{l}$ & $\sigma$ & $D$ & $g$\\
$\left(kg/m^3\right)$ & $\left(kg/m^3\right)$ & $\left(Pa \, s\right)$ 
& $\left(Pa \,s \right)$ & $\left(N/m\right)$ & $(m)$ & $(m /s^{2})$ \\
\hline
1.2 & $0.9982 \times 10^3$ & $1.98 \times 10^{-5}$ & 
$8.9 \times 10^{-4}$ & $0.0728$ & $3 \times 10^{-3}$ & $9.81$\\
\hline\hline
\end{tabular}
\caption{Parameter values used in the simulation 
	of a falling water droplet in air. \label{raindropprop}}
\end{center}
\end{table*}


In order to properly reproduce and analyse the dynamics 
of a relatively large drop (high Reynolds flow) such as in our case, 
the numerical method has to accurately resolve the thin boundary layers 
\sidenote{Even for simulations with 60 points across the droplet diameter, 
the boundary layer has only by 3-4 cells across it. }, 
the interaction of such layers with the capillary forces and finally 
the non-linear feedback of the complex 
3D vortical structures present in the wake behind the droplet. 
Such an undertaking was attempted by Dodd and Ferrante \cite{dodd2014}, 
in which they managed to delineate the different regimes concerning the 
behavior of the wake behind the droplet, although at relatively 
lower Reynolds numbers (the maximum Reynolds tested was $\simeq 500$, 
whereas in our case it is $\simeq 1500$).    
Therefore, our objective behind the demonstration of this particular 
test case is \textit{not} to develop a high fidelity model of a raindrop 
\sidenote{In the Dodd and Ferrante \cite{dodd2014} study , 
a droplet was allowed to fall from rest
along a domain with a length corresponding to 32 times the droplet diameter, 
therefore necessitating a problem size of approximately 260 million cells.}, 
but rather carry out a stringent evaluation of the robustness of our numerical 
method compared to ones that are not mass-momentum consistent. 
For such a low Weber number the capillary forces dominate and the 
droplet should remain intact, and definitely not undergo any subsequent atomization. 


% -----
\begin{figure}[h!]
\begin{center}
\includegraphics[width=0.5\textwidth]{plots/raindrop/setup.png}
\end{center}
\caption{A 2D schematic of the numerical setup for the falling raindrop. 
	A droplet of diameter $D$ is placed at the center of a cubic domain 
	of side $L$ and $L/D = 4$. The liquid properties ($\rho_l$ , $\mu_l$) 
	correspond to that of water, and the gas properties ($\rho_g$,$\mu_g$) 
	correspond to that of air. We apply a uniform inflow velocity condition 
	with $U_0(t)$ and an outflow velocity condition at the top which 
	corresponds to zero normal gradient.
	Boundary conditions on the side walls correspond 
	to those of impenetrable free slip (no shear stress).}
\label{setup}
\end{figure}

%---------------------------------------------------------------
\section{Numerical Instabilities}

Numerical simulations of this configuration at moderate resolution 
\sidenote{Droplet resolutions between $D/h = 16$ to $D/h = 64$ are considered
to be moderate, where $D$ is the droplet diameter and $h$ is the grid size.}
carried out with the standard version
\sidenote{The standard version is based on the non-conservative formulation
of the Navier-Stokes equations, concomitantly not ensuring consistency
between discrete mass and momentum transport.}
of our numerical method results in the catastrophic 
deformations of the droplet as illustrated in Fig. \ref{explode_compare}, 
which we describe as `fictitious' or `artificial' atomization. 

\subsection*{Instability Mechanism}

We propose the following explanation in order to account for such numerical artifacts. 
To start with, we neglect gravity and viscous effects at this relatively large Reynolds number. 
Also, we are interested in steady-state flow 
\sidenote{As the simulation starts with uniform flow in the gas and a zero velocity
in the drop, there is a sudden large acceleration in the gas, resulting in the development 
of a dipolar velocity field akin to that of potential flow around a cylinder (sphere). }. 
On the axis and near the hyperbolic stagnation point 
at the front of the droplet one has $u_2=0$ for the transverse 
(radial) velocity and for the axial momentum balance


\begin{align}
u_1 \partial_1 u_1 = - \frac 1 \rho \partial_1 p.
\end{align}

Due to the large viscosity and density ratios, it is not possible for 
the air flow to immediately entrain the water, so the fluid velocity 
is significantly smaller inside the droplet. 
In the air the acceleration near the stagnation point is 
of the order $U^2/D$, whereas the pressure gradient is

\begin{align}
\partial_1 p \sim \rho_{g} U^2/D.
\end{align}

The pressure gradient in the liquid is much smaller, 
however, in the case of a mixed cell the water density 
multiplies with the gas acceleration $U^2/D$, so that

\begin{align}
\partial_1 p \sim \rho_{l} U^2/D,
\end{align}

then a large pressure gradient results in the mixed cell or cells. 
This large pressure gradient results in a pressure spikes inside the 
droplet near the front stagnation point, as shown in Figure \ref{pressure_1}. 
Such pressure gradients are balanced by surface tension only for a sufficiently 
large curvatures near the droplet front. 
This explains the presence of a ``dimple'' often observed in low 
resolution simulations of the falling drop. 
This artifact had been observed by Xiao \cite{xiao2012} in a similar case
\sidenote{In the thesis of Xiao, the focus was on the analysis of droplet breakup 
corresponding to different Weber number regimes.}
involving the sudden interaction of a droplet at rest with a uniform gas flow. 
The resulting large un-physical pressure gradients across the interface 
eventually lead to its rapid destabilization and concomitant breakup.  

\begin{figure*}
\begin{center}
\begin{tabular}{cc}
\hspace*{-1.0cm}
\includegraphics[width=0.8\textwidth]{plots/raindrop/pressure_compare.png} &
\hspace{-0.4cm}%
\includegraphics[width=0.8\textwidth]{plots/raindrop/nonmc_16ppd_pressure.png}\\
\hspace{-0.8cm}%
(a) & (b)
\end{tabular}
\end{center}
\caption{The origin of the pressure peak in the front of the droplet. 
(a) The profile of the pressure on the axis a few timesteps after initialisation 
with the standard, non-consistent method (red curve) and the consistent method based
on the conservative Navier-Stokes formulation (green curve). 
Much larger pressure gradients are present across the interface using the non-consistent method. 
(b) The pressure distribution immediately after the start of the simulation 
using the standard, non-consistent method. The pressure peak has not yet resulted
in the formation of a dimple. In all figures the droplet resolution corresponds to $D/h = 16$.
The simulations are carried out with the CIAM advection method, in conjunction with the Superbee limiter.}
\label{pressure_1}
\end{figure*}

\begin{figure*}
\begin{center}
\begin{tabular}{cc}
\hspace*{-1.0cm}
\includegraphics[width=0.8\textwidth]{plots/raindrop/ke_compare.png} &
\hspace{-0.4cm}%
\includegraphics[width=0.8\textwidth]{plots/raindrop/mc_16ppd_pressure.png}\\
\hspace{-0.8cm}%
(a) & (b)
\end{tabular}
\end{center}
\caption{ (a) Comparison of the temporal evolution of droplet kinetic energy. 
The standard non-consistent method displays spikes in the kinetic energy that are 
approximately 3 orders of magnitude larger than that of the consistent method, 
leading to rapid destabilization. 
(b) The pressure distribution immediately after the start of the simulation 
using the consistent method (based on a conservative Navier-Stokes formulation). 
In all figures the droplet resolution corresponds to $D/h = 16$.
The simulations are carried out with the CIAM advection method, in conjunction with the Superbee limiter.}
\label{pressure_2}
\end{figure*}

\begin{figure*}
\begin{center}
\begin{tabular}{cc}
\hspace*{-1.0cm}
\includegraphics[width=0.8\textwidth]{plots/raindrop/mc_vorticity_zoom.png} & 
\hspace{-0.4cm}%
\includegraphics[width=0.8\textwidth]{plots/raindrop/mc_vorticity.png} \\ 
\hspace{-0.8cm}%
(a) & (b)
\end{tabular}
\end{center}
\caption{Flow field around the $3 mm$ droplet with $D/h = 16$ immediately 
after the start of the simulation with the consistent method, 
demonstrating the contours of the norm of the vorticity field (black lines) . 
The 2D cross-section in these figures corresponds to the 
mid-plane slice along the Z axis, where the inflow is along 
the X axis and gravity opposite to it. (a) The velocity magnitude. 
As one can observe, the boundary layer is resolved by only 2-3 cells. 
(b) The velocity component in the Y direction, perpendicular to the flow. 
As the flow develops further, a marked separation of the boundary layers is observed with 
a more complex vortical region in the wake.
The figures correspond to simulations carried out with the 
CIAM advection method, in conjunction with the Superbee limiter.}
\label{flow_field}
\end{figure*}


Visualization of the flow around the droplets (Figure \ref{flow_field}) 
illustrates the challenging nature of the flow configuration, 
even for such a seemingly simple physical problem. 
As one can observe, the boundary layers are extremely thin. 
Therefore, even though the velocity field is continuous across the 
interface (in the discrete sense) in the absence of mass transfer, 
there is the appearance of strong velocity variations at the 
scale of the grid size for such coarse levels of grid refinement. 

\begin{figure*}
\begin{center}
\includegraphics[width=1.25\textwidth]{plots/raindrop/raindrop_explode.png}
\end{center}
\caption{A comparison of the temporal evolution between the standard 
non-consistent method (top row) and the consistent method based on the 
conservative formulation of the governing equations (bottom row), 
using the combination of CIAM advection scheme coupled with the Superbee flux limiter. 
The flow is along the positive X direction, with gravity along the opposite direction. 
The red contour indicates the isosurface of the volume fraction 
field corresponding to a value of $0.5$, whereas the black contours 
surrounding the drop represent isosurfaces of the magnitude of vorticity. 
The raindrop with the non-consistent method displays massive deformations 
leading to artificial breakup as a result of rapidly growing numerical instabilities. 
The droplet resolution for both methods is $D/h = 16$ .}
\label{explode_compare}
\end{figure*}


%---------------------------------------------------------------

\section{Stabilization Strategies}

The cascading nature of the numerical instabilities that lead to 
the eventual (un-physical) fragmentation of droplet is not just a 
cause of concern in the context of modeling a raindrop, but it has important
implications within the more broader scope of 
atomization and other complex fragmentation phenomena. 


\subsection*{Primary Concern}
Arguably, the most important objective of numerical investigations of physical
phenomena involving fragmentation is the quantification of the size of 
the features that result from a series of topological changes 
\sidenote{In the context of surface tension dominated flows, these features are 
primarily drops and bubbles, occassionally one can also be 
interested in transient features such as thin sheets, ligaments etc. }. 
A typical example is the statistical distributions of drop sizes, 
where the drops are generated via some fragmentation mechanism. 
\sidenote{These are generally in the form of probability density functions (PDF).}
As demonstrated in the present studies using the standard non-consistent method,  
the artificially atomized drop leads to the generation of a large number of 
smaller fragments, which if counted, will skew the resulting droplet size 
distributions towards the smaller sizes.
Therefore, the falling raindrop serves as a faithful representation of the 
numerous under-resolved features that are typically omnipresent in
numerical studies of complex fragmentation phenomena.  
The inability to discern between the drops resulting from physically consistent 
mechanisms and those that result from `numerical fragmentation' is exactly
why there is a need to develop numerical schemes that circumvent the occurrence
of such numerical instabilities.  



\subsection*{Consistent Mass-Momentum Transport}

The most common and brute force approach that one can apply 
in order to suppress or circumvent such numerical instabilities is 
by using a combination of extremely refined meshes coupled with larger domains
\sidenote{In the study of Dodd and Ferrante \cite{dodd2014}, 
a computational domain with length corresponding to 32 times the diameter is
employed, resulting in a problem size of approximately 260 million cells, 
that too for a relatively low Reynolds number.}
A more computationally efficient approach would be to use a 
conservative formulation
\sidenote{A prerequisite of methods that use consistent mass-momentum transport 
is the formulation of the convective operators in the conservative manner i.e
divergence of fluxes instead of gradients of the primary (discontinuous) variables.} 
of the Navier-Stokes equations, in order
to achieve consistency in the discrete transport of mass and momentum. 
In this section, we shall take a closer look at how the utilisation of 
the consistent method enables us not only to stabilize majority of the numerical
instabilities we have come across, but also delivers relatively good approximations 
of the flow dynamics involved. A thorough and detailed exposition of the principles
behind the consistent method and the different strategies towards achieving consistent
transport is covered in the ensuing chapter.  

We observe that applying the method that ensures consistent mass-momentum 
transport brings a considerable and systematic improvement over a 
spectrum of different flux limiters (WENO, ENO, Superbee, QUICK, Verstappen) 
and CFL numbers, as evidenced by comparing the figures \ref{pressure_1} and \ref{pressure_2}. 
To summarize, the simulations broadly fall into three categories. 
The first two categories concern those that use the method ensuring  
consistent mass-momentum transport, and as a result maintain 
physical values of the kinetic energy and smooth interface shapes. 
Certain simulations amongst the ones carried out with this consistent  
method display some numerical instabilities (but not leading 
to catastrophic deformation)
\sidenote{This may be due certain issues regarding the treatment 
of outflow boundary conditions in our numerical platform `PARIS Simulator'.}
resulting due to the particular choice of 
the advection scheme (CIAM/WY) and 
flux limiter (QUICK/WENO/ENO/Superbee/Verstappen) combination. 

\begin{figure}
\begin{center}
\includegraphics[width = 1.0\textwidth]{plots/raindrop/combined_failures_raindrop.png}
\end{center}
\vspace*{-0.5cm}
\caption{A juxtaposition of the different manifestations of  
`artificial' atomization one encounters while using the standard 
method that does not involve consistency between the discrete transport of mass and momentum. 
The red contours indicate the isosurface of the volume fraction 
field corresponding to a value of $0.5$, acting as a good proxy for the exact interfacial position. 
One can clearly observe that the un-physical fragmentation of the raindrop is symptomatic 
of the non-consistent method, systematically across all combinations of flux limiters and advection schemes. 
The symbol ``WY'' in the legend corresponds to those run using 
the Weymouth-Yue advection scheme, and ``CIAM'' corresponds to the CIAM scheme. 
Brief descriptions of these advection schemes can be found in the preceding chapter \ref{ch:method}.
The implementation of the non-linear flux limiters i.e WENO, ENO, QUICK, Superbee, Verstappen 
correspond to that of well established methods developed to 
deal with hyperbolic conservation laws, for more details refer to the
studies of Leveque \cite{flim_1} and Sweby \cite{flim_2}.
The time stamps are indicative of the moments at which the interface
is the most deformed, and do not necessarily correspond to the moment
at which the code crashes.} 
\label{explode_all}
\end{figure}

\begin{figure*}
\begin{center}
\includegraphics[width = 1.5\textwidth]{plots/raindrop/multiplot_raindrop.png}
\end{center}
\vspace*{-0.5cm}
\caption{Temporal evolution of quantities of interest to evaluate the 
performance of the consistent scheme for different spatial resolutions. 
(a) Kinetic energy of the droplet. 
(b) Moment of inertia of the droplet along the flow (X) direction. 
(c) and (d) Moment of inertia of the droplet along the directions 
perpendicular to flow (Y,Z), evolution of $I_{yy}$ 
seems to be more or less identical to $I_{zz}$.}
\label{multi}
\end{figure*}

Finally, in the third category are the simulations carried out with the standard 
non-consistent method that rapidly blow up, displaying marked peaks 
or spikes in kinetic energy as a function of time, 
associated with massively deformed interface shapes (see figure \ref{explode_compare}). 
Additionally, our studies suggest that certain combinations 
of the advection scheme and the flux limiter are 
numerically more robust than others
\sidenote{Further detailed investigations are required
to understand the difference in code stability as a function
of the choice of flux limiter and advection scheme combination. }
, in particular, the most stable combinations 
are that of CIAM advection with 
Superbee limiter, and the WY advection with QUICK limiter. 

\begin{figure}
\begin{center}
\includegraphics[width = 1.0\textwidth]{plots/raindrop/dropl_velocity_accel_ppd.png}
\end{center}
\vspace*{-0.5cm}
\caption{Comparison of droplet velocity as a function of time, 
for different droplet resolutions.
The simulations were carried out with the consistent version of our method,
using the combination of the WY advection scheme with the QUICK limiter. 
The droplet velocities correspond to that of their respective center of masses. 
Inset : Convergence of the droplet acceleration as a function of resolution, 
computed using the best linear fit over the temporal variation 
of their respective velocities. 
The error bars signify the asymptotic standard 
error (least-squares) corresponding to the obtained linear fits.} 
\label{drop_vel}
\end{figure}

\begin{figure*}
\begin{center}
\begin{tabular}{cc}
\hspace*{-1.0cm}
\includegraphics[width=0.7\textwidth]{plots/raindrop/raindrop_mass_conv_16ppd.png} & 
\hspace{-0.2cm}%
\includegraphics[width=0.7\textwidth]{plots/raindrop/raindrop_mass_conv_clipping.png} \\ 
\hspace{-0.2cm}%
(a) & (b) \\
\hspace*{-1.0cm}
\includegraphics[width=0.7\textwidth]{plots/raindrop/raindrop_mass_conv_resolution.png} & 
\hspace{-0.2cm}%
\includegraphics[width=0.7\textwidth]{plots/raindrop/acc_conv.png} \\ 
\hspace{-0.2cm}%
(c) & (d)
\end{tabular}
\end{center}
\caption{Relative change in the mass of the droplet 
as a function of time in plots (a), (b) and (c). 
The simulations are carried out with a resolution of $D/h = 16$, 
for a total time of $0.1$ milliseconds. 
The symbol ``WY'' in the legend corresponds to those run using 
the Weymouth-Yue advection scheme in combination with the QUICK limiter, 
and ``CIAM'' corresponds to the CIAM scheme in combination with the Superbee limiter. 
(a) The dependence of the mass conservation properties of 
the scheme as a function of the Poisson solver tolerance is tested. 
For the WY-QUICK combination, using a stricter tolerance results 
in better mass conservation, whereas the conservation properties of the 
CIAM-Superbee combination seems to be independent of the tolerance.
(b) Dependence of the mass conservation on the clipping parameter employed. 
The CIAM-Superbee combination again seems to be impervious to changes in clipping, 
whereas WY-QUICK seems to perform slightly better by lowering the parameter.
(c) Dependence of the mass conservation properties on the droplet resolution. 
The WY-QUICK combination displays better results for all except the lowest resolution.   
(d) Variation of the relative error from the least-squares fit on the droplet acceleration, 
in the frame of reference of the static box. 
The corresponding droplet accelerations are plotted in the inset of figure \ref{drop_vel}. 
The reduction in the relative error follows roughly second-order spatial convergence.}
\label{mass_conv}
\end{figure*}

We illustrate the performance of the consistent method through the 
results of our simulations in Figure \ref{multi}. 
The spatial resolutions correspond to $D/h = 8, 16, 32 $ and $64$ 
using the WY advection scheme in combination with the QUICK flux limiter, 
while keeping the same value for the inflow velocity boundary condition. 
The quantities of interest while examining the robustness of the 
method are the temporal evolution of the droplet kinetic energy 
(Fig. \ref{multi}. (a) ) and the moment of inertia of the droplet 
along the three directions (Fig. \ref{multi}. (b),(c),(d)) . 
The inflow is along the X direction with gravity opposite to it. 
The moment of inertia is used as a descriptor of the 'average' droplet shape, 
with the three moments of inertia along the different axes $I_m$ defined as - 


\begin{align}
	I_m = \int_{\mathcal{D}} H x_m^2 d \boldsymbol{x} \quad , \quad \text{where}, \quad 1 \le m \le 3
\end{align}

where $\mathcal{D}$ is the computational domain and $x_m$ is the 
distance of the interface relative to the center of mass of the droplet.   
The kinetic energy of the droplet evolves in a relatively smooth manner, 
without the presence of sudden spikes and falls which are emblematic of 
the standard non-consistent method (refer to Fig. \ref{pressure_2}). 
Such abrupt changes in kinetic energy of the droplet have been 
found to be associated with instants when the droplet undergoes 
`artificial' atomization or breakup, henceforth resulting in 
the catastrophic loss of stability for the numerical method. 
We observe a systematic drop in the amount of the droplet 
kinetic energy as we increase resolution, with the most probable explanation 
being that of the suppression of spurious interfacial 
oscillations which are rampant at lower resolutions. 
There is also a component of the kinetic energy of the droplet 
associated with the internal coherent vortical structures generated due to 
the interaction of aerodynamic shear at the interface, 
evidenced by the non-zero value of the kinetic 
energy even for the most highly resolved droplets. 
Finally, the moment of inertia of the droplet appears to evolve 
in a smooth manner for all droplet resolutions, with higher 
resolutions exhibiting lower amounts of inertia as a result of 
the more compact shapes obtained once the 
spurious interfacial deformation modes are subdued. 
This observation holds true for the moment of inertia curves regardless of direction.           


We have to keep in mind that the velocity inflow condition 
does not correspond to the `exact' teminal velocity 
field an actual falling raindrop might experience, 
hence the droplet in our numerical setup does have some 
finite acceleration due to the imbalance 
between the aerodynamic forces and gravity. 
In Figure \ref{drop_vel}, we demonstrate the velocity of 
the center of mass of the droplet (in the frame of reference 
of the box enclosing the droplet) as a function of time, 
and its behavior as we increase the droplet resolution. 
As one can observe, due to the imbalance of forces acting on the droplet, 
it undergoes a net acceleration as evidenced by the approximately 
linear increase (absolute value) in the velocity as a function of time. 
The temporal variation in the droplet velocity is fitted to a 
linear polynomial in order to evaluate the droplet acceleration 
by means of a standard least-squares approach, for each droplet resolution. 
We illustrate (inset Figure \ref{drop_vel}) that we achieve more 
accurate fits as a consequence of higher droplet resolutions, 
as evidenced by a decrease in the standard error on the fits 
ranging from $\pm 11.1 \%$ for $D/h = 8$ to a value of $\pm 0.12 \%$ 
for the highest resolution of $D/h = 64$. 
The results obtained in this section indicate that especially 
when it comes to low resolutions, 
our numerical method can be used to get relatively good 
estimates of the underlying flow features of the droplet 
without observing an un-physical evolution as a consequence of discretization errors.           


