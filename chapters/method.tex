\setchapterpreamble[u]{\margintoc}
\chapter{Methodology}
\labch{method}

%---------------------------------------------------------------
% overview of chapter 
In this chapter, we describe the basic numerical methodology behind our models 
concerning the dynamics of immiscible liquid-gas interfacial flows, at the incompressible and isothermal limits. 
These implementations are developed on the platforms 'PARIS Simulator' \cite{paris} and 
'Basilisk' \cite{popinetbasilisk}, with considerable overlap between the two platforms in terms 
of the treatment of the interface capturing schemes, transport of conserved quantities and surface tension models.
\sidenote{The principle difference between 'PARIS Simulator' and 'Basilisk' is the ability to resolve the conservation
laws on dynamically adaptive meshes in the case of 'Basilisk', whereas 'PARIS Simulator' only deals with regular Cartesian meshes.}
The numerical implementations are based on finite volume discretizations on uniform or dynamically refined Cartesian grids , utilizing
state of the art methods in interfacial reconstruction coupled with geometric
transport of the corresponding fluxes, curvature computation and surface tension modeling. For more detailed
descriptions of the general capabilities of 'PARIS Simulator’ and 'Basilisk', we refer the reader to the previously cited references. 


\section{Governing Equations}

We use the one-fluid formulation for our system of governing equations, thus solving 
the incompressible Navier-Stokes equations throughout the whole domain including regions 
of variable density and viscosity, which itself depend on the explicit 
location of the interface separating the two fluids.
In the absence of mass tranfer, the velocity field is continuous across
the interface at the incompressible limit, with the interface evolving according to the local velocity vector.  
Generally, we have a choice regarding how to model the convective operator
of the incompressible Navier-Stokes equations. There is a well established corpus of 
numerical methods tailored specifically to deal with non-conservative 
\sidenote{also referred to as the strong form, necessitating certain orders of smoothness of the primitive variable} form of the convective 
operator that appear in transport equations of mass and momentum 
\sidenote{These methods are descendants of the class of numerical schemes used to solve hyperbolic partial differential equations.}
, which perform quite well in the context of single phase flows.
However, in interfacial flows we often deal with discontinuities that arise as a consequence
of the contrast in material properties between the two fluids. Therefore, even though the velocity field
remains continuous throughout the domain, the otherwise smooth density (mass) and momentum fields 
contain sharp jumps (discontinuities) localized at the interfacial position.     
Therefore, we choose to formulate our governing equations in a conservative form i.e involving divergence of fluxes instead of gradients of the primitive variables when it comes to the convective operator. More detailed discussions and analyses about the comparative advantages of the conservative formulation in the context of flows involving large density-ratios is the focus of the subsequent chapters. Thus, the equations are as follows :  



% in incompressible flows, the velocity field is continuous in the absence of mass transfer across the interface 

\begin{align} 
	\frac{\partial \rho}{\partial t} + \nabla\cdot \left(\rho\boldsymbol{u}\right) &= 0 \label{mass} \\
	\frac{\partial}{\partial t} \left(\rho\boldsymbol{u}\right) + \nabla\left(\rho\boldsymbol{u}\otimes\boldsymbol{u}\right)  &= -\nabla p + \nabla \cdot \left(2 \mu \boldsymbol{D}\right) + \sigma \kappa \delta_{s}\boldsymbol{n} + \rho \boldsymbol{g}
\label{nseqn}
\end{align}


with $\rho$ and $\mu$ being the density and dynamical viscosity respectively, and therefore are the physical quantities which are discontinuous across the interface. The volumetric sources are modeled by the acceleration $g$, and the deformation rate tensor $\boldsymbol{D}$ used to model the viscous stresses is defined as:  

\begin{align}
	\boldsymbol{D} = \frac{1}{2}\left[\nabla \boldsymbol{u} + \left(\nabla \boldsymbol{u}\right)^{T}\right]  
\end{align}


The term $\sigma \kappa \delta_{s}\boldsymbol{n}$ models the surface tension forces in the 
framework of the continuum surface-force (CSF) method. The normal vector to the interface 
is $\boldsymbol{n}$, with $\sigma$ being the coefficient of surface tension and $\kappa$ the 
local interfacial curvature. The operator $\delta_{s}$ is the Dirac delta function, 
the numerical approximation of which allows us to map the singular surface force distribution
along the interface onto their volumetric equivalents for our Cartesian control volumes. 
At the incompressible limit, the advection of mass given by equation \ref{mass} can be 
treated as equivalent to that of the advection of volume.


\paragraph{Evolution of phase-characteristic function}
Within the framework of interface capturing schemes, the 
temporal evolution of the interface separating the two fluids
can be modeled by the following advection equation : 

\begin{align} 
	\frac{\partial \chi}{\partial t} + \boldsymbol{u}\nabla\chi = 0 	
\label{chi}
\end{align}


where $\chi$ is the phase-characteristic function. The two main (and the most popular)
approaches in the context of interface capturing schemes are the  
which is mathematically equivalent to a Heaviside function in space and time. This function returns $1$ where the space is occupied by the first fluid, and $0$ for the second fluid. At the macroscopic length scales under consideration, the interface evolution as described by equation \ref{chi} is modeled as having infinitesimal thickness under the continuum hypothesis. The coupling of the interfacial evolution with the equations of fluid motion as described in \ref{nseqn} is provided by :  

% evolution of marker function - marker functiona approximated by LS, VOF etc

% possibility of harmonic mean in play 


\paragraph{Description of Operators}
\blindtext


% also explain how diffusion and capillary force operators are computed, briefly




%---------------------------------------------------------------

\section{Interface Tracking}

\paragraph{Volume-of-Fluid : PLIC Methodology}
\blindtext

\paragraph{Flux Computation : CIAM , WY}
\blindtext

%---------------------------------------------------------------

\section{Time Marching}

\paragraph{Spatio-Temporal Discretization}
\blindtext

\paragraph{Pressure-Projection Algorithm}
\blindtext


